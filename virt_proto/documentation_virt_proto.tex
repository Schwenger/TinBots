\documentclass[a4paper,parskip,headheight=38pt]{scrartcl} % article or scrartcl
\usepackage[utf8]{inputenc}
\usepackage[T1]{fontenc}
\usepackage{amsmath,amssymb,amsfonts}
\usepackage[%
  automark,
  headsepline                %% Separation line below the header
]{scrlayer-scrpage}
\usepackage[english]{babel}
\usepackage{hyphenat}
\usepackage[hidelinks]{hyperref}
\usepackage[top=1.4in, bottom=1.5in, left=1in, right=1in]{geometry}
\usepackage{lastpage}
\usepackage{csquotes}
\usepackage{microtype}
\usepackage{datetime}

\usepackage[normalem]{ulem}
\usepackage{enumerate}
\usepackage{hyperref}

\usepackage{microtype}

\usepackage[hang]{footmisc}
\setlength{\footnotemargin}{3mm}

\usepackage{enumitem}

\usepackage[export]{adjustbox}

% \usepackage{multicol}
\usepackage{graphicx}
\usepackage{graphics}
% \usepackage{float}
% \usepackage{caption}

\parindent 0pt
\parskip 6pt

\clubpenalty = 10000
\widowpenalty = 10000
\displaywidowpenalty = 10000

\renewcommand\thesubsection{\Alph{subsection}}

\setkomafont{pagehead}{\normalfont\sffamily\footnotesize}
\addtolength{\headheight}{+13pt}
\lohead{Marlene Böhmer, s9meboeh@stud.uni-saarland.de, 2547718 \\
	Maximilian Köhl, s8makoeh@stud.uni-saarland.de, 2553525 \\
	Maximilian Schwenger, schwenger@stud.uni-saarland.de, 2542438\\
	Ben Wiederhake, s9bewied@stud.uni-saarland.de, 2541266}
\rohead{\includegraphics[height=36pt, right]{../logo/logo.png} \newline ES16, Specification, Group 6, Page {\thepage}/{\pageref*{LastPage}}}

\newtimeformat{mytime}{\twodigit{\THEHOUR}\twodigit{\THEMINUTE}\twodigit{\THESECOND}}
\settimeformat{mytime}
\newdateformat{mydate}{\twodigit{\THEYEAR}\twodigit{\THEMONTH}\twodigit{\THEDAY}}
\cfoot{\tiny\texttt{ID \mydate\today\currenttime}}
\chead{} % Needed because now the \subsections get displayed
\pagestyle{scrheadings}

% \renewcommand{\headrulewidth}{0pt}
% \addtolength{\textheight}{+30mm}
% \addtolength{\textwidth}{+50mm}
% \addtolength{\hoffset}{-7mm}

% \newcommand{\Omicron}{\ensuremath{\mathcal{O}}}
% \newcommand{\omicron}{\ensuremath{o}}
% \newcommand{\set}[1]{\{#1\}}
% \newcommand{\abs}[1]{\lvert #1 \rvert}

\DeclareMathOperator{\sinc}{sinc}

\begin{document}
We wrote our own library which contains custom blocks that allow us to build simulations. Because of the complexity of the whole system we decided to use a unit test like approach to test specific components.

You will find an overview over all customs blocks in the library file itself:

<FIXME: insert screenshot>

\section{Environment}
Because there is no prebuilt block in Simulink which would be useful for the Environment, we needed to come up with our own approach. The maze and all entities in it are described in a text file which is loaded into MatLab and transformed into a matrix. This matrix represents the entire environment and is used in raycast computations to determine sensor readings of the proximity sensors and IR sensors at a given point in time and space. This happens within our custom sensor blocks.

The motor of the E-Puck is realized by a custom differential drive block implementing the real world physics by using adequate differential equations. The block assures that the Tin Bot cannot drive through other objects by looking up the occupation status in the matrix and restricting the motion if necessary.

The complete physical Tin Bot is represented by the block \enquote{Tin Bot Physical}. This block feeds the differential drive's data (position and orientation) in the appropriate sensor blocks — taking the orientation offset of the sensors into account — such that their values can be computed appropriately using racasting.

Depending on the Tin Bot's current position — as given by the differential drive — the matrix is updated such that Tin Bots cannot drive through each other. Therefore a Tin Bot Introducer block has to be connected to the block representing the physical Tin Bot. This block writes an appropriate value in the matrix at the Tin Bot's current position. To use multiple Tin Bots within the simulation, the Introducer blocks are chained such that at the end there is a matrix containing all the Tin Bots.

The map used by the Tin Bot's sensors however must not contain the particular Tin Bot. Therefore the Tin Bot Eliminator block deletes the particular Tin Bot again from the environment. This prevents the sensors from detecting themselves.

The LPS block takes the environment's data and feeds it into the Tin Bots every two seconds. This data is then used within the software blocks. The LPS block allows the LPS to be in two states, enabled and disabled as described in our specification document.

The output of the control software is feed back to the environment, which allows us to model real world physics.

\section{Components}
Besides the model of the environment, there are various other components mainly modelling the software running on the Tin Bot as Stateflow charts.

<FIXME insert description of the software components>

\section{Test-Cases}
As already mentioned we used a unit test like approach to test all aspects of the system.

\begin{description}
\item[drive\_test] tests the physical model of differential drive (quick, atomic)
\item[proximity\_test] tests the physical model of the proximity sensors (quick, atomic)
\item[ir\_test] tests the physical model of IR sensor (quick, atomic)
\item[traffic\_blind\_test] tests the \enquote{blind} part of the traffic cop (quick, atomic)
\item[victim\_direction\_test] test victim direction calculation
    Medium time, tests environment and bare-bone software component
\item[path\_finder\_test] test internal map generation and escape path calculation
    Medium time, awesome visualization
\item[approximator\_test] test approximation of current position and orientation
    Slow, complete test of internal position and orientation approximation based on the
    motor speeds and reset of wrong data according to the LPS (crash into wall)
\item[follow\_right\_hand\_test] test right hand rule
    Very slow, nearly holistic
\item[controller\_test] tests the complete control software, and use cases \textbf{UC-E} and \textbf{UC-F} (very slow, nearly holistic)
\end{description}

<TODO: add a manual switch to change the map (use case D)>

<TODO: implement use case C by bypassing the map chain>

Because of the complexity of the system we did not implement use cases \textbf{UC-A} and \textbf{UC-B} explicitly. However you may think of them as starting and stopping the simulation.

\end{document}
