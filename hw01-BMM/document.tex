\documentclass[a4paper,parskip,headheight=38pt]{scrartcl} % article or scrartcl
\usepackage[utf8]{inputenc}
\usepackage{amsmath,amssymb,amsfonts}
\usepackage[%
  automark,
  headsepline                %% Separation line below the header
]{scrlayer-scrpage}
\usepackage[english]{babel}
\usepackage{hyphenat}
\usepackage[hidelinks]{hyperref}
\usepackage[top=1.4in, bottom=1.5in, left=1in, right=1in]{geometry}
\usepackage{lastpage}
\usepackage{csquotes}
\usepackage{microtype}
\usepackage{datetime}

\usepackage[normalem]{ulem}
\usepackage{enumerate}
\usepackage{hyperref}

% \usepackage{multicol}
\usepackage{graphicx}
\usepackage{graphics}
% \usepackage{float}
% \usepackage{caption}

\setkomafont{pagehead}{\normalfont\sffamily\footnotesize}
\addtolength{\headheight}{+6pt}
\lohead{Marlene Böhmer, s9meboeh@\ldots, 25xxxxx \\
Markus Köhler, mail@koehlma.de, 25xxxxx \\
Ben Wiederhake, s9bewied@\ldots, 2541266}
\rohead{\newline \newline ES16, Set 1, Page {\thepage}/{\pageref*{LastPage}}}

\newtimeformat{mytime}{\twodigit{\THEHOUR}\twodigit{\THEMINUTE}\twodigit{\THESECOND}}
\settimeformat{mytime}
\newdateformat{mydate}{\twodigit{\THEYEAR}\twodigit{\THEMONTH}\twodigit{\THEDAY}}
\cfoot{\tiny\texttt{ID \mydate\today\currenttime}}
\chead{} % Needed because now the \subsections get displayed
\pagestyle{scrheadings}

% \renewcommand{\headrulewidth}{0pt}
% \addtolength{\textheight}{+30mm}
% \addtolength{\textwidth}{+50mm}
% \addtolength{\hoffset}{-7mm}

% \newcommand{\Omicron}{\ensuremath{\mathcal{O}}}
% \newcommand{\omicron}{\ensuremath{o}}
% \newcommand{\set}[1]{\{#1\}}
% \newcommand{\abs}[1]{\lvert #1 \rvert}

\begin{document}

\section*{Problem 1: Simulink}

\subsection*{Part 1: Sketch}

\includegraphics[width=\textwidth]{p1-sketch}

\subsection*{Part 2: Differential equations}

\begin{align*}
    v(0) &= 5 \\
    dv/dt &= -10 \\
    s(0) &= 20 \\
    ds/dt &= v(t)
\end{align*}


\section*{Problem 2: Car Motion on a Slope}

In order to be an equilibrium, the state must have $\dot{x} = 0 \land
\dot{v} = 0$, which implies $v=0$ and (with $v=0$ already applied) $-(m g
\sin \theta) / m = 0$.

One class of equilibria is $g=0, v=0$.  Note that this \enquote{class}
contains infinitely many different states, as we didn't specify $x$.

If one defines $\frac{0}{0} = 0$, then $m=0, v=0$ is another such class.

Since $\theta$ and $F$ are given in the exercise, and can't be changed,
there are no other equilibria.


\section*{Problem 3: Stability of Equilibria}

Reasoning like above:
    %
\begin{align*}
    \text{equilibrium}
    &\implies \dot{s}_1 = 0 \land \dot{s}_2 = 0 \\
    &\implies 3s_1 + 4s_2 = 0 \land 2s_1 + s_2 = 0 \\
    &\implies 3s_1 + 4(-2s_1) = 0 \land 2s_1 + s_2 = 0 \\
    &\implies s_1 = 0 \land s_2 = 0
\end{align*}

So there can only be one equilibrium (as there is no further
dimension).  It can also easily be seen that this candidate is in fact
an equilibrium.

Consider the state $(s_1=\varepsilon, s_2 = 0)$.  Obviously, this
quickly evolves towards $(s_1 \to \infty, s_2 \to \infty)$.  Therefore,
it is unstable (= not stable and not asymptotically stable).


\section*{Problem 4: Cyclist}

\subsection*{Part 1}

Here's the model we created:

\includegraphics[width=\textwidth]{p4a-proof}

We chose $v_{0} = 0$ and $s_{0} = 0$ as starting conditions for both
integrations, as this piece of information is missing in the exercise
text.

\subsection*{Part 2}

We created an appropriate file and ran a simulation with a sufficiently
small fixed-size step.

 \pagebreak{}
Our \texttt{.mat} file can be reproduced like this (in Matlab itself):
    %
\vspace{-\baselineskip}\begin{verbatim}
myts = timeseries([]) % matrix-based bags don't support non-linear time
myts = addsample(myts, 'Data', 0, 'Time', 0)
myts = addsample(myts, 'Data', 0, 'Time', 2)
myts = addsample(myts, 'Data', 1, 'Time', 2)
myts = addsample(myts, 'Data', 1, 'Time', 7)
myts = addsample(myts, 'Data', 0, 'Time', 7)
myts = addsample(myts, 'Data', 0, 'Time', 8)
save data -v7.3 myts
\end{verbatim}

The 100m mark seems to be reached at exactly 14.5 seconds with our
starting conditions.  In maths: $f(14.5) = 100$

\includegraphics[width=\textwidth]{p4b-proof-small}


\section*{Problem 5: Oscillator}

\subsection*{Part 1}

We believe 103 is closer to the answer than 104.

\includegraphics[width=\textwidth]{p5-proof}

\subsection*{Part 2}

\includegraphics[width=\textwidth]{p5b-model}

No function blocks were added.  The old connection
\enquote{Displacement (y) $\rightarrow$ Gain (k)} was severed, in favor
of the following new function blocks:
    %
\begin{itemize}
    \item \enquote{inner\_paren}, which represents the result of $(y(t)
    - \frac{1}{k}u(t))$.  This function block has the parameter
    \enquote{\texttt{+-}} for correct signs.
        %
    \item \enquote{Gain (1/k)}, which represents the result of
    $\frac{1}{k}u(t))$.  This function block has the parameter $0.1$,
    as that's the value of $\frac{1}{k}$.
        %
    \item \enquote{Cosine}, which is actually a \enquote{Sine
    Wave}-block, with the parameters amplitude=1,
    frequency=$0.5*(2\pi)$, and phase=$pi/2$.
\end{itemize}


\section*{Problem 6: Geiger-Müller Counter}

FIXME


\end{document}
