\documentclass[a4paper,parskip,headheight=38pt]{scrartcl} % article or scrartcl
\usepackage[utf8]{inputenc}
\usepackage[T1]{fontenc}
\usepackage{amsmath,amssymb,amsfonts}
\usepackage[%
  automark,
  headsepline                %% Separation line below the header
]{scrlayer-scrpage}
\usepackage[english]{babel}
\usepackage{hyphenat}
\usepackage[hidelinks]{hyperref}
\usepackage[top=1.4in, bottom=1.5in, left=1in, right=1in]{geometry}
\usepackage{lastpage}
\usepackage{csquotes}
\usepackage{microtype}
\usepackage{datetime}

\usepackage{todonotes}

\usepackage[normalem]{ulem}
\usepackage{enumerate}
\usepackage{hyperref}

\usepackage{microtype}

\usepackage[hang]{footmisc}
\setlength{\footnotemargin}{3mm}

\usepackage{enumitem}

\usepackage{color}

\usepackage[export]{adjustbox}

% \usepackage{multicol}
\usepackage{graphicx}
\usepackage{graphics}
% \usepackage{float}
% \usepackage{caption}

\parindent 0pt
\parskip 6pt

\clubpenalty = 10000
\widowpenalty = 10000
\displaywidowpenalty = 10000

\setkomafont{pagehead}{\normalfont\sffamily\footnotesize}
\addtolength{\headheight}{+13pt}
\lohead{Marlene Böhmer, s9meboeh@stud.uni-saarland.de, 2547718 \\
	Maximilian Köhl, s8makoeh@stud.uni-saarland.de, 2553525 \\
	Maximilian Schwenger, schwenger@stud.uni-saarland.de, 2542438\\
	Ben Wiederhake, s9bewied@stud.uni-saarland.de, 2541266}
\rohead{\includegraphics[height=36pt, right]{../logo/logo.png} \newline ES16, Verification Document, Group 6, Page {\thepage}/{\pageref*{LastPage}}}

\newtimeformat{mytime}{\twodigit{\THEHOUR}\twodigit{\THEMINUTE}\twodigit{\THESECOND}}
\settimeformat{mytime}
\newdateformat{mydate}{\twodigit{\THEYEAR}\twodigit{\THEMONTH}\twodigit{\THEDAY}}
\cfoot{\tiny\texttt{ID \mydate\today\currenttime}}
\chead{} % Needed because now the \subsections get displayed
\pagestyle{scrheadings}

% \renewcommand{\headrulewidth}{0pt}
% \addtolength{\textheight}{+30mm}
% \addtolength{\textwidth}{+50mm}
% \addtolength{\hoffset}{-7mm}

% \newcommand{\Omicron}{\ensuremath{\mathcal{O}}}
% \newcommand{\omicron}{\ensuremath{o}}
% \newcommand{\set}[1]{\{#1\}}
% \newcommand{\abs}[1]{\lvert #1 \rvert}

\DeclareMathOperator{\sinc}{sinc}

\newcommand{\incomplete}[1]{\textless{} #1 \textgreater{}}

\newcommand{\teststrat}[5]{
	\textbf{Component:} #1 \\
	\noindent\textbf{What should be tested?} \\
    \noindent #2 \\
	\noindent\textbf{How can it be tested?} \\
    \noindent\textcolor{blue}{Setup:} #3 \\
    \noindent\textcolor{blue}{Applicable Techniques:} #4 \\
	\noindent\textbf{What cannot be tested? Why?} \\
    \noindent #5
}

\newcommand{\ie}{i.e.}

\begin{document}

\section{Overview}

On a grand scheme the software is divided into the following subcomponents:

\begin{itemize}
	\item Controller
	\item Traffic Cop Eyes
	\item Blind Traffic Cop 
	\item RHR
	\item Victim Direction
	\item Victim Finder
	\item Path Finder
	\item Path Executor
	\item Pickup Artist
\end{itemize}

Each of those components is subject of one or more test strategies either for
the whole component at once, or their subcomponents separately.
\subsection{Controller}
\subsubsection{Purpose}
	Connects each component's inputs and outputs. 
Implementation: Done. 
Testing: On Demand. Whenever another subcomponent is finished and ready to be tested in an integration test, the respective part of the controller is tested extensively as well.

Traffic Cop Eyes:
Connects sensor data with remaining components. Issues requests for other components to start their work.
Implementation: Done.
Testing: Implicit. Successfully used in MatLab, on E-Puck tested in terms of other unit's tests.

Blind Traffic Cop:
Basically an arbiter for the subcomponents.
Implementation: Done.
Testing: Implicit. Closely related to controller, a bug becomes very obvious based on the E-Puck's reacting which is a direct result of the logic-free controller.

RHR:
Implementation of the right hand rule.
Implementation: Done. Addendum: In addition to the virtual prototype, we also cured a stroke, i.e., the faulty behavior that the E-Puck's left half was ignored.
Testing: Done. Extensive integration tests. Tin Bot ran in circles around the maze for hours straight. Problematic factor is lighting conditions. Taken path visualized in web interface.

Victim Direction:
Turns the E-Puck and computes angle/position to the victim.
Implementation: Done.
Testing: Done. In MatLab, Unit tests, as well as on the actual E-Puck. Problems: Reflecting caused by the struts. Very prone to reflections by nearby gazers or changing lighting conditions. Solution: Using cellular rubber (thanks to Peter!) to absorb rather than reflect the signal. Solution is tested in parts.

Victim Finder: 
Starts victim direction. Manages gathered data about the victim, i.e. discards old information and start computation for new data accordingly.
Implementation: Done / in debug phase. 
Testing: Unit tests done, MatLab running, on actual E-Puck unknown, assumed to be fine, though.

Path Finder: 
Computes a path from the current position to the victim based on the internal map consisting of data gathered via proximity sensors and other Tin Bot's broadcasts.
Implementation: State Machine done, search algorithm will be changed from A* to Bellman-Ford.
Testing: State Machine and A* are tested via unit tests, Bellman-Ford is about to be implemented and will be subject to appropriate unit tests. Memory overhead has been considered for both algorithms and static guarantees allow us to deem the algorithm correct in this respect. On MatLab: causes simulation to crash.

Path Executor:
Executes path given by path finder.
Implementation: done.
Testing: Untested.

Pickup Artist:
New component. Activated after path executor reports that the victim has been reached. Decides what actions to take to assure that we can pick up the victim securely and initialize the motion outwards.
Implementation: Not yet.
Testing: Untested.

Moreover, regarding communication, the data (de-)serialization is fully implemented and tested, both for communication with the LPS, and T2T.

\section{Hardware} 
The hardware is divides into two parts: The environment, and the connection
between the main components, \ie, the Raspberry Pi, and the E-Puck. 
The connection between the components and their respective sensors and actuators
is verified using the following test strategies.

\subsection{Connection Extension Board to E-Puck}

\teststrat{Custom Extension Board Connection Frequency}{
    We guarantee a real time upper bound on the frequency in which sensor data
    is received and can then be processed. The time between two received
    sensor data packages is \todo[inline]{FREQUENCY}ms.

    We test the frequency in which data is received and computed the maximal
    time needed for all interrupt service routine to run through, such that the
    stated frequency can be guaranteed.
}{
    Oscilloscope, custom extension board, E-Puck.
}{
    We directly measure the signals from the extension board sent using the
    i\textsuperscript{2}c protocol. We send signals to an IR-Sensor to see
    whether the received signal is correct.
}{
   We do not measure whether an amplitude is registered only if the signal is
   strong enough. The reason for this is that we regulate the signals stability
   on with software, therefore we do not mind the actual intensity. Moreover, we
   designed the IR-Sensors in a way that the sensitivity fits our needs based on
   empirical results.
}

\subsection{Connection between the Raspberry Pi as LPS and the E-Puck as Tin Bot
via Bluetooth}

\teststrat{LPS Update Frequency}{
    If the initial setup is completed, the LPS is supposed to send location data
    to the Tin Bot approximately every two seconds without any hard guarantees.
    However, the process of receiving and processing the data should not take
    more than three seconds. Progress in this regard is supposed to be signaled
    by toggling a designated LED.
}{
    LPS, debug monitor, Tin Bot with a designated LED.
}{
    Make sure the initial setup is completed, \ie, the debug monitor indicates a
    Bluetooth connection and the LPS has recognized the Tin Bot. Now, the time
    between toggles is measured and should not exceed three seconds.
}{
    The correctness of the LPS data is not tested. We assured that the data is
    correct in separate tests, though. Moreover, the location data has to be
    correct relative to the measured location data of other Tin Bots, which
    depends on the camera and is robust by construction.
}

\subsection{Victim}

\teststrat{Victim Actuators}{
    The victim shall send IR-Signals and a LED shall be on after turning the
    victim on. The signals shall be in agreement with the SOS-protocol.
}{
    Oscilloscope, Victim, Victim's power LED, IR-Sensors.
}{
    Turn on the victim and check the LED using visual feedback. Make sure that
    there is a clear sight between the victim and the IR-Sensors. Check the
    received signal using the oscilloscope and compare against the expected
    pattern.
}{
    We cannot test whether the signal originates at the victim or any other
    source, e.g.\ a malevolent agent turning on or off a nearby device using a
    similar signal.
}


All hardware
components are built. Communication protocols are
implemented and extensively tested. Empirical tests in terms of any other E-Puck
based test. - Communication frequency:    + I2C: Test case lighting LEDs after a
poll has been taken place.   + Bluetooth (to LPS): LED indicates when new data
has been received and processed.   + Bluetooth (T2T): Actual communication
untested.

The border has to be coated in cellular rubber, other than that, no further actions have to be applied.


\end{document}