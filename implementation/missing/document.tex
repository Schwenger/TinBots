\documentclass[a4paper,parskip,headheight=38pt]{scrartcl} % article or scrartcl
\usepackage[utf8]{inputenc}
\usepackage[T1]{fontenc}
\usepackage{amsmath,amssymb,amsfonts}
\usepackage[%
  automark,
  headsepline                %% Separation line below the header
]{scrlayer-scrpage}
\usepackage[english]{babel}
\usepackage{hyphenat}
\usepackage[hidelinks]{hyperref}
\usepackage[top=1.4in, bottom=1.5in, left=1in, right=1in]{geometry}
\usepackage{lastpage}
\usepackage{csquotes}
\usepackage{microtype}
\usepackage{datetime}

\usepackage[normalem]{ulem}
\usepackage{enumerate}
\usepackage{hyperref}

\usepackage{microtype}

\usepackage[hang]{footmisc}
\setlength{\footnotemargin}{3mm}

\usepackage{enumitem}

\usepackage[export]{adjustbox}

% \usepackage{multicol}
\usepackage{graphicx}
\usepackage{graphics}
% \usepackage{float}
% \usepackage{caption}

\parindent 0pt
\parskip 6pt

\clubpenalty = 10000
\widowpenalty = 10000
\displaywidowpenalty = 10000

\renewcommand\thesubsection{\Alph{subsection}}

\setkomafont{pagehead}{\normalfont\sffamily\footnotesize}
\addtolength{\headheight}{+13pt}
\lohead{Marlene Böhmer, s9meboeh@stud.uni-saarland.de, 2547718 \\
	Maximilian Köhl, s8makoeh@stud.uni-saarland.de, 2553525 \\
	Maximilian Schwenger, schwenger@stud.uni-saarland.de, 2542438\\
	Ben Wiederhake, s9bewied@stud.uni-saarland.de, 2541266}
\rohead{\includegraphics[height=36pt, right]{../../logo/logo.png} \newline ES16, Prototype shortcomings, Group 6, Page {\thepage}/{\pageref*{LastPage}}}

\newtimeformat{mytime}{\twodigit{\THEHOUR}\twodigit{\THEMINUTE}\twodigit{\THESECOND}}
\settimeformat{mytime}
\newdateformat{mydate}{\twodigit{\THEYEAR}\twodigit{\THEMONTH}\twodigit{\THEDAY}}
\cfoot{\tiny\texttt{ID \mydate\today\currenttime}}
\chead{} % Needed because now the \subsections get displayed
\pagestyle{scrheadings}

% \renewcommand{\headrulewidth}{0pt}
% \addtolength{\textheight}{+30mm}
% \addtolength{\textwidth}{+50mm}
% \addtolength{\hoffset}{-7mm}

% \newcommand{\Omicron}{\ensuremath{\mathcal{O}}}
% \newcommand{\omicron}{\ensuremath{o}}
% \newcommand{\set}[1]{\{#1\}}
% \newcommand{\abs}[1]{\lvert #1 \rvert}

\DeclareMathOperator{\sinc}{sinc}

\begin{document}

\section{E-Puck not as specified}

We are short on time as we had to deal with a few problems with the
E-Puck library and E-Puck specification:
 %
\begin{itemize}
    \item Proximity sensors do not yield values as specified.  We had
    to measure the raw values, come up with a reasonable approximating
    formula, which took time that we now lack for other development.
 %
    \item The Bluetooth-handler of the E-Puck firmware does not have a
    nice way to prevent packet loss, and does not support packages over
    64 bytes (which is far too low for our use case).  This took time
    that we now lack for other development.
 %
    \item The E-Puck firmware uses several different interrupts, and it
    was highly unclear which interrupt timers were used by the firmware
    or free for usage by us.  This took time that we now lack for other
    development.
\end{itemize}

Our approach was to create out own firmware, which (so far) proves to
be more usable and less error-prone than the pre-existing one.

\section{T2T communication}

Technically, Tin-Bot-to-Tin-Bot-communication only improves the global
completeness of internal maps, despite being a major technical effort.
Therefore, this feature is \emph{not} present in our prototype.  Thus
we will present a \emph{single}\footnote{thus we also don't need any
synchronization that makes sure that precisely one Tin Bot rescues the
victim} working Tin Bot.
 \\
Although T2T is already planned for and in some parts already
specified, our approach is to implement T2T last.  (However, of course
we already use Bluetooth to exchange debugging data and the internal
map, so we effectively \emph{are} implementing T2T as we go anyway.)

\section{Rescue mode}

We only have two options for presentation: either we demo the
Right-Hand-Follower (which definitely works), or we demo the overall
system, which would require \emph{all} the components to be perfectly
working.  At least currently, the next stage does not work yet (the
\texttt{victim\_direction} module needs working IR sensors; see below).
We can't guarantee the latter for the prototype yet, so we only demo
the rescue mode RHR.
 \\
For the final product, our approach is to get all components working.

\section{Moving the victim} % VICTOR

Although our current prototype of the victim is very light and offers
very little friction against the ground, the Tin Bot can't turn while
the victim is attached.  This is bad, as our atomic path-following
block first turns and then moves.
 \\
For the final product, our approach is to solve this by any of the
following (whichevery works, arbitrary order):
 %
\begin{itemize}
    \item making the victim even more light-weight
 %
    \item changing the turning building-block to move forward-backward
    in order to turn; not unlike a car
 %
    \item using another E-Puck for the victim, which then actively
    follows the \enquote{rescuer}
\end{itemize}

\section{IR sensors (victim detection)}

As it turns out, our IR sensors are by far not working as well as one
could have expected from IR remotes (we already are using the same
basic principles: quick pulses on a high-frequency carrier-band) and
our previous experiments with our own IR sensors (which work as
intended with a single \enquote{screen} of duct tape).
 \\
This seems to be due to excessive unexpected IR reflections of the
environment, highly directed IR emitters.
 \\
Our approach is to sacrifice angular and temporal resolution in return
for stability.

\section{What \emph{does} work}

Right-Hand-Following definitely works, so this is part of the
prototype.  The LPS also works, including image recognition, conversion
to the required units and sending the data via Bluetooth.  As
Right-Hand-Following does not make use of this information, the LPS is
demoed through a web interface instead.

\section{What is implemented but not presented}

Most internal logic and path finding is already implemented and
unit-tested.  However, there is no really visible intermediate step
between Right-Hand-Following and truly solving the maze, so we can't
show off the internal logic (all the state machines in the
"Controller", as per virtual Prototype).
 \\
Our approach for the Prototype Session will be to try and get certain
parts working in stand-alone mode anyway.

\end{document}
