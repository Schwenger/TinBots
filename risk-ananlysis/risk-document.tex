\documentclass[a4paper,parskip,headheight=38pt]{scrartcl} % article or scrartcl
\usepackage[utf8]{inputenc}
\usepackage[T1]{fontenc}
\usepackage{amsmath,amssymb,amsfonts}
\usepackage[%
  automark,
  headsepline                %% Separation line below the header
]{scrlayer-scrpage}
\usepackage[english]{babel}
\usepackage{hyphenat}
\usepackage[hidelinks]{hyperref}
\usepackage[top=1.4in, bottom=1.5in, left=1in, right=1in]{geometry}
\usepackage{lastpage}
\usepackage{csquotes}
\usepackage{microtype}
\usepackage{datetime}

\usepackage[normalem]{ulem}
\usepackage{enumerate}
\usepackage{hyperref}

\usepackage{microtype}

\usepackage[hang]{footmisc}
\setlength{\footnotemargin}{3mm}

\usepackage{enumitem}

\usepackage[export]{adjustbox}

% \usepackage{multicol}
\usepackage{graphicx}
\usepackage{graphics}
% \usepackage{float}
% \usepackage{caption}

\parindent 0pt
\parskip 6pt

\clubpenalty = 10000
\widowpenalty = 10000
\displaywidowpenalty = 10000

\setkomafont{pagehead}{\normalfont\sffamily\footnotesize}
\addtolength{\headheight}{+13pt}
\lohead{Marlene Böhmer, s9meboeh@stud.uni-saarland.de, 2547718 \\
	Maximilian Köhl, s8makoeh@stud.uni-saarland.de, 2553525 \\
	Maximilian Schwenger, schwenger@stud.uni-saarland.de, 2542438\\
	Ben Wiederhake, s9bewied@stud.uni-saarland.de, 2541266}
\rohead{\includegraphics[height=36pt, right]{../logo/logo.png} \newline ES16, Risk Document, Group 6, Page {\thepage}/{\pageref*{LastPage}}}

\newtimeformat{mytime}{\twodigit{\THEHOUR}\twodigit{\THEMINUTE}\twodigit{\THESECOND}}
\settimeformat{mytime}
\newdateformat{mydate}{\twodigit{\THEYEAR}\twodigit{\THEMONTH}\twodigit{\THEDAY}}
\cfoot{\tiny\texttt{ID \mydate\today\currenttime}}
\chead{} % Needed because now the \subsections get displayed
\pagestyle{scrheadings}

% \renewcommand{\headrulewidth}{0pt}
% \addtolength{\textheight}{+30mm}
% \addtolength{\textwidth}{+50mm}
% \addtolength{\hoffset}{-7mm}

% \newcommand{\Omicron}{\ensuremath{\mathcal{O}}}
% \newcommand{\omicron}{\ensuremath{o}}
% \newcommand{\set}[1]{\{#1\}}
% \newcommand{\abs}[1]{\lvert #1 \rvert}

\DeclareMathOperator{\sinc}{sinc}

\newcommand{\incomplete}[1]{\textless{} #1 \textgreater{}}

\begin{document}

\section{Basic Components}
\subsection{Tin Bot}
\subsubsection{Power LED}
\begin{description}
\item[Failure Rate:]
\item[Source:]
\end{description}
\subsubsection{Front LED}
\subsubsection{Victim Indicator LEDs}
\subsubsection{Power Switch}
\subsubsection{IR-Sensors}
\subsubsection{Proximity Sensors}
\subsubsection{ATmega328}
\subsubsection{E-Puck Controller Board}
\subsubsection{E-Puck Motors and Driver Board}
\subsubsection{I2C Bus}

\subsection{LPS}
\subsubsection{Raspberry}
\subsubsection{Camera}
\subsubsection{USB Connection}
\subsubsection{LPS Software}

\subsection{Victim}
\subsubsection{Battery}
\subsubsection{ATtiny2313}
\subsubsection{IRLZ44N}
\subsubsection{IR-LEDs}
\subsubsection{Controller Software}

\incomplete{FIXME}


\section{Fault Trees and Computed Failure Probabilities}

Please note that due to the complex interactions of the
four\footnote{LPS, E-Puck, extension board, victim} autonomous systems,
our fault tree\footnote{technically: fault DAG, as some intermediate
failures share cause fault/failure} is very large.  Thus, it is not
meaningful to write a paragraph about each intermediate failure.
Instead, we chose to leave out \emph{some} of the intermediate
failures.

Please note that \emph{all} the top-level failures and \emph{all} basic
components are listed.

% Example doesn't use any numeration, so we don't, either.
\subsection*{\incomplete{Our favourite top-level failure}}

\incomplete{FIXME}


\subsection*{Other top-level failures}

\incomplete{FIXME}


\end{document}
