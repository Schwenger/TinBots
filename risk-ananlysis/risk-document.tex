\documentclass[a4paper,parskip,headheight=38pt]{scrartcl} % article or scrartcl
\usepackage[utf8]{inputenc}
\usepackage[T1]{fontenc}
\usepackage{amsmath,amssymb,amsfonts}
\usepackage[%
  automark,
  headsepline                %% Separation line below the header
]{scrlayer-scrpage}
\usepackage[english]{babel}
\usepackage{hyphenat}
\usepackage[hidelinks]{hyperref}
\usepackage[top=1.4in, bottom=1.5in, left=1in, right=1in]{geometry}
\usepackage{lastpage}
\usepackage{csquotes}
\usepackage{microtype}
\usepackage{datetime}

\usepackage[normalem]{ulem}
\usepackage{enumerate}
\usepackage{hyperref}

\usepackage{microtype}

\usepackage[hang]{footmisc}
\setlength{\footnotemargin}{3mm}

\usepackage{enumitem}

\usepackage[export]{adjustbox}

% \usepackage{multicol}
\usepackage{graphicx}
\usepackage{graphics}
% \usepackage{float}
% \usepackage{caption}

\parindent 0pt
\parskip 6pt

\clubpenalty = 10000
\widowpenalty = 10000
\displaywidowpenalty = 10000

\setkomafont{pagehead}{\normalfont\sffamily\footnotesize}
\addtolength{\headheight}{+13pt}
\lohead{Marlene Böhmer, s9meboeh@stud.uni-saarland.de, 2547718 \\
	Maximilian Köhl, s8makoeh@stud.uni-saarland.de, 2553525 \\
	Maximilian Schwenger, schwenger@stud.uni-saarland.de, 2542438\\
	Ben Wiederhake, s9bewied@stud.uni-saarland.de, 2541266}
\rohead{\includegraphics[height=36pt, right]{../logo/logo.png} \newline ES16, Risk Document, Group 6, Page {\thepage}/{\pageref*{LastPage}}}

\newtimeformat{mytime}{\twodigit{\THEHOUR}\twodigit{\THEMINUTE}\twodigit{\THESECOND}}
\settimeformat{mytime}
\newdateformat{mydate}{\twodigit{\THEYEAR}\twodigit{\THEMONTH}\twodigit{\THEDAY}}
\cfoot{\tiny\texttt{ID \mydate\today\currenttime}}
\chead{} % Needed because now the \subsections get displayed
\pagestyle{scrheadings}

% \renewcommand{\headrulewidth}{0pt}
% \addtolength{\textheight}{+30mm}
% \addtolength{\textwidth}{+50mm}
% \addtolength{\hoffset}{-7mm}

% \newcommand{\Omicron}{\ensuremath{\mathcal{O}}}
% \newcommand{\omicron}{\ensuremath{o}}
% \newcommand{\set}[1]{\{#1\}}
% \newcommand{\abs}[1]{\lvert #1 \rvert}

\DeclareMathOperator{\sinc}{sinc}

\newcommand{\incomplete}[1]{\textless{} #1 \textgreater{}}

\begin{document}

\section{Basic Components}

\subsection{Tin Bot} % ===================================

\subsubsection{Power LED}
Indicator LED for power.
\begin{description}
	\item[Failure Rate:] $\leq 10^{-5}$ per hour
	\item[Source:] Luminaire Lifetime report\footnote{\url{http://apps1.eere.energy.gov/buildings/publications/pdfs/ssl/led\_luminaire-lifetime-guide\_june2011.pdf}}
\end{description}

\subsubsection{Front LED}
Front LED on E-Puck.
\begin{description}
	\item[Failure Rate:] $\leq 10^{-5}$ per hour
	\item[Source:] Luminaire Lifetime report
\end{description}

\subsubsection{Victim Indicator LEDs}
Red LEDs on E-Puck.
\begin{description}
	\item[Failure Rate:] $\leq 10^{-5}$ per hour
	\item[Source:] Luminaire Lifetime report
\end{description}

\subsubsection{Power Switch}
E-Puck's power switch.
\begin{description}
	\item[Failure Rate:] $\leq 10^{-6}$ per hour
	\item[Source:] The failure rate is estimated to be low as the power switch is a simple standard component.
\end{description}

\subsubsection{IR-Sensors}
IR-sensors on the extension board.
\begin{description}
	\item[Failure Rate:] $\leq 10^{-5}$ per hour
	\item[Source:] Estimated to be similar to the LEDs.
\end{description}

\subsubsection{Proximity Sensors}
The E-Puck's proximity sensors.
\begin{description}
	\item[Failure Rate:] $\infty$
	\item[Source:] The proximity sensors seem not to work as advertised.
	\item[Experiment Setup:] We sent the raw proximity sensor data via Bluetooth to a smartphone and saw that it clearly does not follow the specification: the sensor values decrease with the increasing distance and not even the operating range applies.
	\item[Desired Failure Rate:] $\leq 10^{-1}$ per hour
	\item[Source:] We have to do some more experiments and hope to achieve a better failure rate. For now we estimate a high failure rate of $\leq 10^{-1}$ to be able to give a result.
\end{description}

\subsubsection{ATmega328}
Processor controlling the signals picked up from the victim.
\begin{description}
	\item[Failure Rate:] $\leq 10^{-12}$ per hour
	\item[Source:] Atmel datasheet\footnote{\url{http://www.atmel.com/images/Atmel-8271-8-bit-AVR-Microcontroller-ATmega48A-48PA-88A-88PA-168A-168PA-328-328P\_datasheet\_Complete.pdf}}
\end{description}

\subsubsection{E-Puck Controller Board}
The E-Puck's controller.
\begin{description}
	\item[Failure Rate:] $\leq 10^{-10}$ per hour
	\item[Source:] The failure rate is estimated to be slightly higher than the one of ATmega328 because it has more components.
\end{description}

\subsubsection{E-Puck Motors and Driver Board}
The E-Puck's motors and motor controller.
\begin{description}
	\item[Failure Rate:] $\leq 10^{-5}$ per hour
	\item[Source:] The failure rate of the E-Puck's motors and motor controller is estimated to be low as it is a standard component and the E-Pucks are not used a lot.
\end{description}

\subsubsection{E-Puck Memory}
The E-Puck's memory.
\begin{description}
	\item[Failure Rate:] $\leq 10^{-11}$ per hour
	\item[Source:] We assume that the E-Puck memory is comparable to the ATmega328's memory\footnote{\url{http://www.atmel.com/images/Atmel-8271-8-bit-AVR-Microcontroller-ATmega48A-48PA-88A-88PA-168A-168PA-328-328P\_datasheet\_Complete.pdf}}\footnote{\url{https://www.wolframalpha.com/input/?i=\%281+hour\%29\%2F\%282e7+years\%29}}
\end{description}

\subsubsection{I2C Bus}
Cable connecting the controlling E-Puck and the E-Puck's extension board.
\begin{description}
	\item[Failure Rate:] $\leq 10^{-5}$ per hour
	\item[Source:] The failure rate is estimated to be low due to the simple construction.
\end{description}

\subsubsection{Magnets}
The magnets attached to the E-Puck.
    % Both positive and negative failures
\begin{description}
	\item[Failure Rate:] $\leq 10^{-5}$ per hour
	\item[Source:] The chance that the magnets trigger unintentionally, i.e. through a wall, is low since the Tin Bot has to drive close a section of a wall while the victim's location coincidentally is close to this very wall.
\end{description}

\subsubsection{Bluetooth Module}
The E-Puck's bluetooth module.
\begin{description}
	\item[Failure Rate:] $\leq 10^{-5}$ per hour
	\item[Source:] The failure rate of the bluetooth module is estimated to be low as it is a standard component and the distances between the connected participants are small.
    \footnote{Example report about Bluetooth being reliable:
    \url{http://www.connectblue.com/fileadmin/Connectblue/Web2006/Documents/White\_papers/Bluetooth\_Reliability.pdf}}
\end{description}

\subsection{Tin Bot Software} % ===================================

\subsubsection{Victim Recognition}
Module that determines whether we're carrying a victim.
\begin{description}
\item[Failure Rate:] $\leq 10^{-3}$ per hour
 %
\item[Source:] As this component uses an educated guess about angles,
size, and sensor sensitivity, this is estimated to be somewhat
unreliable.
 %
\end{description}

\subsubsection{Victim Triangulation}
 %
Module that triangulates the victim's position based on two or more
previous measurements.
 %
\begin{description}
\item[Failure Rate:] 0
 %
\item[Source:] Assuming correct implementation and correct automatic translation
to target hardware, the implementation is correct.
 %
\end{description}

\subsubsection{Avoidance Watchdog}
 %
Module that shall interrupt when we're about to crash into a wall.
 %
\begin{description}
\item[Failure Rate:] 0
 %
\item[Source:] Assuming correct implementation and correct automatic translation
to target hardware, the implementation is correct.
 %
\end{description}

\subsubsection{Crash Watchdog: Right-Hand Follower}
 %
The part of the Right-Hand Follower that tries to avoid crashing into obstacles.
 %
\begin{description}
\item[Failure Rate:] $\leq 10^{-1}$ per hour
 %
\item[Source:] We ran our virtual prototype of the Right-Hand Follower
for 1 simulated hour and did not observe any crash.
 %
\end{description}

\subsubsection{Crash Watchdog: Path Finder/Executor}
 %
The part of the Path Finder/Executor that tries to avoid crashing into obstacles.
 %
\begin{description}
\item[Failure Rate:] $\leq 10^{-2}$ per hour
 %
\item[Source:] We ran our virtual prototype of the Path Finder/Executor
for 1 simulated hour (with hand-chosen destinations upon completion) and
did not observe any crash.  We assume that the failure rate is smaller
than that of the Right-Hand Follower as the update and checking
mechanism is more sophisticated.
 %
\end{description}

\subsubsection{Path Pruning}
 %
Module that shall detect and deter when RHR is about to walk in circles.
 %
\begin{description}
\item[Failure Rate:] 0
 %
\item[Source:] Assuming correct implementation and correct automatic translation
to target hardware, the implementation is correct.
 %
\end{description}

\subsection{LPS} % ===================================

\subsubsection{Raspberry}
General Raspberry Pi module.
\begin{description}
\item[Failure Rate:] $\leq 10^{-6}$ per hour
\item[Source:] We approximate the reliability lower than the one of an FPGA ($10^{-8}$)\footnote{\url{http://www.xilinx.com/support/documentation/user_guides/ug116.pdf}}, but greater than the one of a Camera ($10^{-4}$).
\end{description}

\subsubsection{Camera}
Takes photo and converts it into the RAW format.
\begin{description}
\item[Failure Rate:] $\leq 10^{-4}$ per hour
\item[Source:] Educated guess based on the component's complexity.
\end{description}

\subsubsection{LPS Software (Ours)}
\begin{description}
\item[Failure Rate:] $0$
\item[Source:] Based on the fact that the software is only a loop
executing very simple code, we estimate the code is correct.
\end{description}

\subsubsection{LPS Software (External Module)}
\begin{description}
\item[Failure Rate:] $0$
\item[Source:] We want to use well established libraries and perform rather simple computations, thus we estimate this to be correct.
\end{description}

\subsubsection{LPS Operating System (Linux)}
The Raspberry Pi runs a Linux kernel and performs communications, as well as image processing tasks.
\begin{description}
\item[Failure Rate:] $\leq 2.94 \times 10^{-5}$ per hour
\item[Source:] The IBM Linux Technology Center conducted a 30 day stress test resulting in a success rate of 97.88\%\footnote{\url{https://www.ibm.com/developerworks/library/l-rel/}}.
\end{description}

\subsubsection{Power Supply}
\begin{description}
\item[Failure Rate:] $\leq 10^{-6}$
\item[Source:] Since the Raspberry Pi is widely used for long running processes such as server installation, we can safely assume that the power supply is rather reliable, thus we guess the failure rate educatedly.
\end{description}

\subsubsection{Electricity Grid}
\begin{description}
\item[Failure Rate:] $\leq 2.25 \times 10^{-5}$ %0,00002492389658
\item[Source:] Based on data by the Verband der Elektrotechnik Elektronik Informationstechnik (VDE) in 2014\footnote{\url{http://www.energiezukunft.eu/netze/netzausbau/deutsches-stromnetz-ist-aeusserst-zuverlaessig-gn103605/}}. 
We cannot assume that at the point of deployment there is no major environmental catastrophe taking place since we simulate a rescue team.
\end{description}

\subsection{Victim} % ===================================

\subsubsection{Battery}
Main powers source of the Victim.
\begin{description}
\item[Failure Rate:] $\leq 10^{-8}$ per hour
\item[Source:] Based on the fact that this is a very simple component the probability was estimated.
\end{description}

\subsubsection{ATtiny2313}
Processor controlling light emission of the victim.
\begin{description}
\item[Failure Rate:] $\leq 10^{-12}$ per hour
\item[Source:] Atmel datasheet for this microcontroller\footnote{\url{http://www.atmel.com/images/doc8246.pdf}}
\end{description}

\subsubsection{MOS-FET (IRLZ44N)}
Power amplification for IR-diodes.
\begin{description}
\item[Failure Rate:] $\leq 10^{-12}$ per hour
\item[Source:] Based on the fact that this is a very simple component probability was estimated.
\end{description}

\subsubsection{Resistor(s)}
\begin{description}
\item[Failure Rate:] $\leq 10^{-9}$ per hour
\item[Source:] http://www.sqconline.com/resistor-failure-rate-model-mil-hdbk-217-rev-f-noetice-2
\end{description}

\subsubsection{IR-LEDs}
\begin{description}
\item[Failure Rate:] $\leq 10^{-5}$ per hour
\item[Source:] Luminaire Lifetime report\footnote{\url{http://apps1.eere.energy.gov/buildings/publications/pdfs/ssl/led\_luminaire-lifetime-guide\_june2011.pdf}}, as it is assumed that IR LEDs (which use the same mechanism to emit photons) exhibit the same fault rate.
\end{description}

\subsubsection{Controller Software}
\begin{description}
\item[Failure Rate:] $0$
\item[Source:] Based on the fact that the software is only a loop
executing very simple code, we estimate the code is correct.
\end{description}

\subsubsection{Magnet Belt}
\begin{description}
\item[Failure Rate:] $\leq 10^{-1}$ per hour
\item[Source:] Unreliable construction.
\end{description}


\subsection{Environment} % ===================================

\subsubsection{IR Medium}
 %
Measures how well the environment permits or perturbs an IR signal.
This also cares about noise and background \enquote{luminancy}.
 %
\begin{description}
\item[Failure Rate:] $\leq 10^{-4}$ per hour
 %
\item[Source:] As we assume that there is not much noise and background
luminancy, this should be reasonably small, and the IR signal of the
victim should remain mostly undisturbed by the environment.
 %
\end{description}

\subsubsection{Bluetooth Medium}
 %
Measures how well the environment permits or perturbs Bluetooth communication.
This also cares about noise and background \enquote{luminancy}.
 %
\begin{description}
\item[Failure Rate:] $\leq 10^{-3}$ per hour
 %
\item[Source:] As we assume that there is not much noise and background
luminancy, this should be reasonably small, and Bluetooth communication
of the victim should remain mostly undisturbed by the environment.
 %
\end{description}

 \pagebreak
\subsection*{Computed Failure Probabilities}

We chose $t = 1h$ for the computation of the failure probabilites, so
we have $F = 1-e^{-\lambda}$ each.  As many failure rates are repeated,
here is a conversion table:

%%%%%%  RAW VALUES
%0: 0.0
%0.1: 0.09516258196404043
%0.01: 0.009950166250831947
%1e-3: 0.0009995001666250085
%1e-4: 9.999500016666251e-05
%1e-05: 9.999950000166666e-06
%2.25e-05: 2.2499746876898428e-05
%2.94e-05: 2.9399567824235332e-05
%1e-06: 9.999995000001667e-07
%1e-09: 9.999999995e-10
%1e-10: 9.999999999500001e-11
%1e-11: 9.999999999949999e-12
%1e-12: 9.999999999995e-13

%%%%%%  UNTRUCATED/UNROUNDED VALUES
%\[\begin{array}{rr}
%0 & 0.0 \\
%0.1 & 0.09516258196404043 \\
%1\times 10^{-2} & 0.009950166250831947 \\
%1\times 10^{-3} & 0.0009995001666250085 \\
%1\times 10^{-4} & 9.999500016666251\times 10^{-5} \\
%1\times 10^{-5} & 9.999950000166666\times 10^{-6} \\
%2.25\times 10^{-5} & 2.2499746876898428\times 10^{-5} \\
%2.94\times 10^{-5} & 2.9399567824235332\times 10^{-5} \\
%1\times 10^{-6} & 9.999995000001667\times 10^{-7} \\
%1\times 10^{-9} & 9.999999995\times 10^{-10} \\
%1\times 10^{-10} & 9.999999999500001\times 10^{-11} \\
%1\times 10^{-11} & 9.999999999949999\times 10^{-12} \\
%1\times 10^{-12} & 9.999999999995\times 10^{-13} \\
%\end{array}\]

%%%%%%  ROUNDED VALUES (same rounding as in the trees)
\[\begin{array}{rr}
0.1                 & 0.095163 \\
1\times 10^{-2}     & 0.0099502 \\
1\times 10^{-3}     & 0.0009995 \\
1\times 10^{-4}     & 9.9995\times 10^{-5} \\
1\times 10^{-5}     & 1\times 10^{-5} \\
2.25\times 10^{-5}  & 2.25\times 10^{-5} \\
2.94\times 10^{-5}  & 2.94\times 10^{-5} \\
1\times 10^{-6}     & 1\times 10^{-6} \\
1\times 10^{-8}     & 1\times 10^{-8} \\
1\times 10^{-9}     & 1\times 10^{-9} \\
1\times 10^{-10}     & 1\times 10^{-10} \\
1\times 10^{-11}     & 1\times 10^{-11} \\
1\times 10^{-12}     & 1\times 10^{-12} \\
0                   & 0.0
\end{array}\]

Note that we used a built-in function \texttt{expm1} to get more accurate
results than the naive approach.


\section{Fault Trees and Computed Failure Probabilities}

\subsection*{Top-Level Failures}

Note that MR2, MR4, MR6, MR16, MR18, and MR19 do
not have fault trees, as this concept does not apply here.  Also note
that MR11 has been removed.

\newcommand{\refpdf}[1]{(\texttt{\href{trees/#1.pdf}{\texttt{#1.pdf}}})}

\begin{description}
\item[\texttt{power LED is off} \refpdf{nopowerled}:]
    The power LED is not lit. \\
    This means that either there is a failure in the E-Puck, or the power LED is faulty.
\item[\texttt{E-Puck failure}:]
    There is a problem with the E-Puck. 
    Either the user forgot to turn on the E-Puck or there is a problem with the hardware, such as a problem with the power, the board, or the memory.
\item[\texttt{power failure}:]
    The battery can be faulty, e.g. due to a leakage, or not charged. Additionally, the connection between the battery and its \textit{consumer} might be impaired since a problem with the power switch prevents a connection, or there is a failure in the wiring.
\item[\texttt{uncooperative} \refpdf{uncooperative}:]
    The fault tree of behavior that appears to be uncooperative (MR3, MR22). \\
    The reason for such behavior is that the E-Puck does not know that there are other E-Puck or it did not receive any data to take into account because there was a problem with the communication protocol (T2T). 
    Alternatively, the transportation medium might prevent a meaningful communication.

    A problem with the receiver or the sender can also be the culprit, caused by a software bug, or a problem with the Bluetooth module. % we do not describe the subtrees separately, in regard to BW's and MS's conversation with HT
\item[\texttt{not using information about the victim} \refpdf{ignorevictim}:]
    The E-Puck ignores gathered information about the victim (MR5, MR13). \\
    Potential reasons for this behavior is that the extension board is faulty (see below), the triangulation failed, meaning that we effectively do not have meaningful data about the victim, there are inconsistencies in the data about the victim, or the E-Puck does not consume received information fast enough, such that it gets overwritten.
\item[\texttt{triangulation fails}:]
    Unless there is a software failure due to an unhandled edge case, the triangulation does not provide data because the E-Puck needs to travel a certain distance until a triangulation is triggered. 
    This distance is an empirically determined value and can be inappropriate. 
    Note that this distance is the length of a straight line orthogonal to the line of measure of older triangulations.
\item[\texttt{conflicting data about victim}:]
    Conflicting data about the victim can be the result of an unexpected movement by the victim, i.e. it is moved by the user. 
    Alternatively, uncooperative behavior might lead to conflicting data, since Tin Bots believe other Tin Bots trust, even if they send wrong data. 
    Additionally, faulty memory causes the same behavior.
\item[\texttt{run into walls} \refpdf{runintowall}:]
    A Tin Bot runs into a wall.\\
    The problem is either bad firmware, including that the wrong program has been uploaded to the E-Puck, or that the collision avoidance failed.
\item[\texttt{collision with obstacle}:]
    A collision can be the result of a problem in collision avoidance, or a failure to detect the obstacle.
\item[\texttt{obstacle not detected}:]
    Undetected obstacles are a result of walls that are not in conformity to the specification, e.g. hovering in the air or too small. 
    Otherwise, a software failure or faulty proximity sensors can result in ignoring obstacles.
\item[\texttt{avoidance system fails}:]
    The avoidance system can only fail if any of the drivers fails to avoid a collision on its on \emph{and} the dedicated collision avoidance watchdog fails, too.
\item[\texttt{victimlost} \refpdf{victimlost}:]
    The fault tree of losing the victim during escort (MR7, MR8).
\item[\texttt{noescort} \refpdf{noescort}:]
    The fault tree of (apparently) not escorting the victim properly
    (MR8, MR17). TODO
\item[\texttt{escortnoled} \refpdf{escortnoled}:]
    The fault tree of not flashing properly during escort (MR9). TODO
\item[\texttt{victim404} \refpdf{victim404}:]
    Fault tree of not finding the victim (MR10). TODO
\item[\texttt{victimsilent} \refpdf{victimsilent}:]
    Fault trees of not finding the victim (MR10). TODO
\item[\texttt{seenoled} \refpdf{seenoled}:]
    The fault tree of the IR visualization missing (MR12, MR20, MR21). TODO
\item[\texttt{standingstill} \refpdf{standingstill}:]
    The fault tree of standing still (MR13). TODO
\item[\texttt{gowrong} \refpdf{gowrong}:]
    The fault tree of going to the wrong location at a specific step
    during the rescue action (MR14). TODO
\item[\texttt{runintowall} \refpdf{runintowall}:]
    The fault tree of colliding with a wall or unintentionally with a
    victim (MR15). TODO
\item[\texttt{spuriousmovements} \refpdf{spuriousmovements}:]
    The fault tree of spurious or unreasonable movements (MR23 and
    several others). TODO
\item[\texttt{system failure} \refpdf{systemfailure}:]
    The overall fault tree of the victim not being moved out.  Note
    that this can only happen if \emph{all} Tin Bots fail to move the
    victim out, and is thus a disjunction of the following, previously
    mentioned top-level failures that are fatal to the Tin Bot's
    operability:
     \\
    \texttt{victim cannot be found}, \texttt{not using information
    about the victim}, \texttt{not moving the victim out} and
    \texttt{victim lost while escorting}
\end{description}


\end{document}
