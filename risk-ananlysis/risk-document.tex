\documentclass[a4paper,parskip,headheight=38pt]{scrartcl} % article or scrartcl
\usepackage[utf8]{inputenc}
\usepackage[T1]{fontenc}
\usepackage{amsmath,amssymb,amsfonts}
\usepackage[%
  automark,
  headsepline                %% Separation line below the header
]{scrlayer-scrpage}
\usepackage[english]{babel}
\usepackage{hyphenat}
\usepackage[hidelinks]{hyperref}
\usepackage[top=1.4in, bottom=1.5in, left=1in, right=1in]{geometry}
\usepackage{lastpage}
\usepackage{csquotes}
\usepackage{microtype}
\usepackage{datetime}

\usepackage[normalem]{ulem}
\usepackage{enumerate}
\usepackage{hyperref}

\usepackage{microtype}

\usepackage[hang]{footmisc}
\setlength{\footnotemargin}{3mm}

\usepackage{enumitem}

\usepackage[export]{adjustbox}

% \usepackage{multicol}
\usepackage{graphicx}
\usepackage{graphics}
% \usepackage{float}
% \usepackage{caption}

\parindent 0pt
\parskip 6pt

\clubpenalty = 10000
\widowpenalty = 10000
\displaywidowpenalty = 10000

\setkomafont{pagehead}{\normalfont\sffamily\footnotesize}
\addtolength{\headheight}{+13pt}
\lohead{Marlene Böhmer, s9meboeh@stud.uni-saarland.de, 2547718 \\
	Maximilian Köhl, s8makoeh@stud.uni-saarland.de, 2553525 \\
	Maximilian Schwenger, schwenger@stud.uni-saarland.de, 2542438\\
	Ben Wiederhake, s9bewied@stud.uni-saarland.de, 2541266}
\rohead{\includegraphics[height=36pt, right]{../logo/logo.png} \newline ES16, Risk Document, Group 6, Page {\thepage}/{\pageref*{LastPage}}}

\newtimeformat{mytime}{\twodigit{\THEHOUR}\twodigit{\THEMINUTE}\twodigit{\THESECOND}}
\settimeformat{mytime}
\newdateformat{mydate}{\twodigit{\THEYEAR}\twodigit{\THEMONTH}\twodigit{\THEDAY}}
\cfoot{\tiny\texttt{ID \mydate\today\currenttime}}
\chead{} % Needed because now the \subsections get displayed
\pagestyle{scrheadings}

% \renewcommand{\headrulewidth}{0pt}
% \addtolength{\textheight}{+30mm}
% \addtolength{\textwidth}{+50mm}
% \addtolength{\hoffset}{-7mm}

% \newcommand{\Omicron}{\ensuremath{\mathcal{O}}}
% \newcommand{\omicron}{\ensuremath{o}}
% \newcommand{\set}[1]{\{#1\}}
% \newcommand{\abs}[1]{\lvert #1 \rvert}

\DeclareMathOperator{\sinc}{sinc}

\newcommand{\incomplete}[1]{\textless{} #1 \textgreater{}}

\begin{document}

\section{Basic Components}
\subsection{Tin Bot} % ===================================

\subsubsection{Power LED}
\incomplete{FIXME}

\subsubsection{Front LED}
\incomplete{FIXME}

\subsubsection{Victim Indicator LEDs}
\incomplete{FIXME}

\subsubsection{Power Switch}
\incomplete{FIXME}

\subsubsection{IR-Sensors}
\incomplete{FIXME}

\subsubsection{Proximity Sensors}
\incomplete{FIXME}

\subsubsection{ATmega328}
\incomplete{FIXME}

\subsubsection{E-Puck Controller Board}
\incomplete{FIXME}

\subsubsection{E-Puck Motors and Driver Board}
\incomplete{FIXME}

\subsubsection{E-Puck Memory}
\incomplete{FIXME}

\subsubsection{I2C Bus}
\incomplete{FIXME}

\subsubsection{Magnets}
    % Both positive and negative failures
\incomplete{FIXME}

\subsubsection{Bluetooth Module}
\incomplete{FIXME}

\subsection{Tin Bot Software} % ===================================

\subsubsection{Victim Recognition}
Module that determines whether we're carrying a victim.
\begin{description}
\item[Failure Rate:] $\leq 10^{-3}$ per hour
 %
\item[Source:] As this component uses an educated guess about angles,
size, and sensor sensitivity, this is estimated to be somewhat
unreliable.
 %
\end{description}

\subsubsection{Victim Triangulation}
 %
Module that triangulates the victim's position based on two or more
previous measurements.
 %
\begin{description}
\item[Failure Rate:] 0
 %
\item[Source:] Assuming correct implementation and correct automatic translation
to target hardware, the implementation is correct.
 %
\end{description}

\subsubsection{Avoidance Watchdog}
 %
Module that shall interrupt when we're about to crash into a wall.
 %
\begin{description}
\item[Failure Rate:] 0
 %
\item[Source:] Assuming correct implementation and correct automatic translation
to target hardware, the implementation is correct.
 %
\end{description}

\subsubsection{Path Pruning}
 %
Module that shall detect and deter when RHR is about to walk in circles.
 %
\begin{description}
\item[Failure Rate:] 0
 %
\item[Source:] Assuming correct implementation and correct automatic translation
to target hardware, the implementation is correct.
 %
\end{description}

\subsubsection{Crash: Right-Hand Follower}
 %
Ability of the Right-Hand Follower to collide with an obstacle.
 %
\begin{description}
\item[Failure Rate:] $\infty$
 %
\item[Source:] As we decided against making each Module crash-safe, the
Reliability is 0 (and thus the Failure Rate is $\infty$).
 %
\end{description}

\subsubsection{Crash: Path Finder/Executor}
 %
Ability of the Path Finder/Executor to collide with an obstacle.
 %
\begin{description}
\item[Failure Rate:] $\infty$
 %
\item[Source:] As we decided against making each module crash-safe, the
Reliability is 0 (and thus the Failure Rate is $\infty$).
 %
\end{description}

\subsection{LPS} % ===================================

\subsubsection{Raspberry}
General Raspberry Pi module.
\begin{description}
\item[Failure Rate:] $\approx 10*^{-6}$ per hour
\item[Source:] We approximate the reliability lower than the one of an FPGA ($10^{-8}$)\footnote{\url{http://www.xilinx.com/support/documentation/user_guides/ug116.pdf}}, but greater than the one of a Camera ($10^{-4}$).
\end{description}

\subsubsection{Camera}
Takes photo and converts it into the RAW format.
\begin{description}
\item[Failure Rate:] $\approx 10*^{-4}$ per hour % 0.0000294
\item[Source:] Educated guess based on the component's complexity.
\end{description}

\subsubsection{LPS Software (Ours)}
\begin{description}
\item[Failure Rate:] $0$ per hour
\item[Source:] Based on the fact that the software is only a loop
executing very simple code, we estimate the code is correct.
\end{description}

\subsubsection{LPS Software (External Module)}
\begin{description}
\item[Failure Rate:] $0$ per hour
\item[Source:] We want to use well established libraries and perform rather simple computations, thus we estimate this to be correct.
\end{description}

\subsubsection{LPS Operating System (Linux)}
The Raspberry Pi runs a Linux kernel and performs communications, as well as image processing tasks.
\begin{description}
\item[Failure Rate:] $\approx 2.94 * 10*^{-5}$ per hour % 0.0000294
\item[Source:] The IBM Linux Technology Center conducted a 30 day stress test resulting in a success rate of 97.88\%\footnote{\url{https://www.ibm.com/developerworks/library/l-rel/}}.
\end{description}

\subsubsection{Power Supply}
\begin{description}
\item[Failure Rate:] $10^{-6}$ 
\item[Source:] Since the Raspberry Pi is widely used for long running processes such as server installation, we can safely assume that the power supply is rather reliable, thus we guess the failure rate educatedly.
\end{description}

\subsubsection{Electricity Grid}
\begin{description}
\item[Failure Rate:] $\approx 2.25 * 10^{-5}$ %0,00002492389658
\item[Source:] Based on data by the Verband der Elektrotechnik Elektronik Informationstechnik (VDE) in 2014\footnote{\url{http://www.energiezukunft.eu/netze/netzausbau/deutsches-stromnetz-ist-aeusserst-zuverlaessig-gn103605/}}. We cannot assume that at the point of deployment there is no major environmental catastrophe taking place since we simulate a rescue team.
\end{description}

\subsection{Victim} % ===================================

\subsubsection{Battery}
Main powers source of the Victim.
\begin{description}
\item[Failure Rate:] $\leq 10^{-8}$ per hour
\item[Source:] Based on the fact that this is a very simple component the probability was estimated.
\end{description}

\subsubsection{ATtiny2313}
Processor controlling light emission of the E-Puck.
\begin{description}
\item[Failure Rate:] $\leq 10^{-12}$ per hour
\item[Source:] http://www.atmel.com/images/doc8246.pdf
\end{description}

\subsubsection{MOS-FET (IRLZ44N)}
Power amplification for IR-diodes.
\begin{description}
\item[Failure Rate:] $\leq 10^{-12}$ per hour
\item[Source:] Based on the fact that this is a very simple component probability was estimated.
\end{description}

\subsubsection{Resistor(s)}
\begin{description}
\item[Failure Rate:] $\leq 10^{-9}$ per hour
\item[Source:] http://www.sqconline.com/resistor-failure-rate-model-mil-hdbk-217-rev-f-noetice-2
\end{description}

\subsubsection{IR-LEDs}
\begin{description}
\item[Failure Rate:] $\leq 10^{-5}$ per hour
\item[Source:] http://apps1.eere.energy.gov/buildings/publications/pdfs/ssl/led\_luminaire-lifetime-guide\_
june2011.pdf
\end{description}

\subsubsection{Controller Software}
\begin{description}
\item[Failure Rate:] $0$ per hour
\item[Source:] Based on the fact that the software is only a loop
executing very simple code, we estimate the code is correct.
\end{description}

\subsubsection{Magnet Belt}
\begin{description}
\item[Failure Rate:] $\leq 10^{-1}$ per hour
\item[Source:] Doubtful self made construction.
\end{description}


\subsection{Environment} % ===================================

\subsubsection{IR Medium}
 %
Measures how well the environment permits or perturbs an IR signal.
This also cares about noise and background \enquote{luminancy}.
 %
\begin{description}
\item[Failure Rate:] $10^{-4}$ per hour
 %
\item[Source:] As we assume that there is not much noise and background
luminancy, this should be reasonably small, and the IR signal of the
victim should remain mostly undisturbed by the environment.
 %
\end{description}

\subsubsection{Bluetooth Medium}
 %
Measures how well the environment permits or perturbs Bluetooth communication.
This also cares about noise and background \enquote{luminancy}.
 %
\begin{description}
\item[Failure Rate:] $10^{-3}$ per hour
 %
\item[Source:] As we assume that there is not much noise and background
luminancy, this should be reasonably small, and Bluetooth communication
of the victim should remain mostly undisturbed by the environment.
 %
\end{description}


\section{Fault Trees and Computed Failure Probabilities}

\subsection*{Other Top-Level Failures}

Note that MR2, MR4, MR6, MR16, MR18, and MR19 do
not have fault trees, as this concept does not apply here.  Also note
that MR11 has been removed.

\begin{description}
\item[\texttt{power LED is off}: (see nopowerled.pdf)]
    The power LED is not lit. \\
    This means that either there is a failure in the E-Puck, or the power LED is faulty.
\item[\texttt{E-Puck failure}:]
    There is a problem with the E-Puck. 
    Either the user forgot to turn on the E-Puck or there is a problem with the hardware, such as a problem with the power, the board, or the memory.
\item[\texttt{power failure}:]
    The battery can be faulty, e.g. due to a leakage, or not charged. Additionally, the connection between the battery and its \textit{consumer} might be impaired since a problem with the power switch prevents a connection, or there is a failure in the wiring.
\item[\texttt{uncooperative}: (see uncooperative.pdf)]
    The fault tree of behavior that appears to be uncooperative (MR3, MR22). \\
    The reason for such behavior is that the E-Puck does not know that there are other E-Puck or it did not receive any data to take into account because there was a problem with the communication protocol (T2T). 
    Alternatively, the transportation medium might prevent a meaningful communication.

    A problem with the receiver or the sender can also be the culprit, caused by a software bug, or a problem with the Bluetooth module. % we do not describe the subtrees separately, in regard to BW's and MS's conversation with HT
\item[\texttt{not using information about the victim}: (see ignorevictim.pdf)]
    The E-Puck ignores gathered information about the victim (MR5, MR13). \\
    Potential reasons for this behavior is that the extension board is faulty (see below), the triangulation failed, meaning that we effectively do not have meaningful data about the victim, there are inconsistencies in the data about the victim, or the E-Puck does not consume received information fast enough, such that it gets overwritten.
\item[\texttt{triangulation fails}:]
    Unless there is a software failure due to an unhandled edge case, the triangulation does not provide data because the E-Puck needs to travel a certain distance until a triangulation is triggered. 
    This distance is an empirically determined value and can be inappropriate. 
    Note that this distance is the length of a straight line orthogonal to the line of measure of older triangulations.
\item[\texttt{victimlost}:]
    The fault tree of losing the victim during escort (MR7, MR8).
\item[\texttt{noescort}:]
    The fault tree of (apparently) not escorting the victim properly
    (MR8, MR17).
\item[\texttt{escortnoled}:]
    The fault tree of not flashing properly during escort (MR9).
\item[\texttt{victim404}, \texttt{victimsilent}:]
    Fault trees of not finding the victim (MR10).
\item[\texttt{seenoled}:]
    The fault tree of the IR visualization missing (MR12, MR20, MR21).
\item[\texttt{standingstill}:]
    The fault tree of standing still (MR13).
\item[\texttt{gowrong}:]
    The fault tree of going to the wrong location at a specific step
    during the rescue action (MR14).
\item[\texttt{runintowall}:]
    The fault tree of colliding with a wall or unintentionally with a
    victim (MR15).
\item[\texttt{spuriousmovements}:]
    The fault tree of spurious or unreasonable movements (MR23 and
    several others).
\item[\texttt{systemfailure}:]
    The overall fault tree of the victim not being moved out.  Note
    that this is essentially the \texttt{OR} of some\footnote{e.g.\
    \texttt{standingstill}, but not \texttt{seenoled}} other fault
    trees.
\end{description}


\end{document}
