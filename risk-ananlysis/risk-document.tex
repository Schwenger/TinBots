\documentclass[a4paper,parskip,headheight=38pt]{scrartcl} % article or scrartcl
\usepackage[utf8]{inputenc}
\usepackage[T1]{fontenc}
\usepackage{amsmath,amssymb,amsfonts}
\usepackage[%
  automark,
  headsepline                %% Separation line below the header
]{scrlayer-scrpage}
\usepackage[english]{babel}
\usepackage{hyphenat}
\usepackage[hidelinks]{hyperref}
\usepackage[top=1.4in, bottom=1.5in, left=1in, right=1in]{geometry}
\usepackage{lastpage}
\usepackage{csquotes}
\usepackage{microtype}
\usepackage{datetime}

\usepackage[normalem]{ulem}
\usepackage{enumerate}
\usepackage{hyperref}

\usepackage{microtype}

\usepackage[hang]{footmisc}
\setlength{\footnotemargin}{3mm}

\usepackage{enumitem}

\usepackage[export]{adjustbox}

% \usepackage{multicol}
\usepackage{graphicx}
\usepackage{graphics}
% \usepackage{float}
% \usepackage{caption}

\parindent 0pt
\parskip 6pt

\clubpenalty = 10000
\widowpenalty = 10000
\displaywidowpenalty = 10000

\setkomafont{pagehead}{\normalfont\sffamily\footnotesize}
\addtolength{\headheight}{+13pt}
\lohead{Marlene Böhmer, s9meboeh@stud.uni-saarland.de, 2547718 \\
	Maximilian Köhl, s8makoeh@stud.uni-saarland.de, 2553525 \\
	Maximilian Schwenger, schwenger@stud.uni-saarland.de, 2542438\\
	Ben Wiederhake, s9bewied@stud.uni-saarland.de, 2541266}
\rohead{\includegraphics[height=36pt, right]{../logo/logo.png} \newline ES16, Risk Document, Group 6, Page {\thepage}/{\pageref*{LastPage}}}

\newtimeformat{mytime}{\twodigit{\THEHOUR}\twodigit{\THEMINUTE}\twodigit{\THESECOND}}
\settimeformat{mytime}
\newdateformat{mydate}{\twodigit{\THEYEAR}\twodigit{\THEMONTH}\twodigit{\THEDAY}}
\cfoot{\tiny\texttt{ID \mydate\today\currenttime}}
\chead{} % Needed because now the \subsections get displayed
\pagestyle{scrheadings}

% \renewcommand{\headrulewidth}{0pt}
% \addtolength{\textheight}{+30mm}
% \addtolength{\textwidth}{+50mm}
% \addtolength{\hoffset}{-7mm}

% \newcommand{\Omicron}{\ensuremath{\mathcal{O}}}
% \newcommand{\omicron}{\ensuremath{o}}
% \newcommand{\set}[1]{\{#1\}}
% \newcommand{\abs}[1]{\lvert #1 \rvert}

\DeclareMathOperator{\sinc}{sinc}

\newcommand{\incomplete}[1]{\textless{} #1 \textgreater{}}

\begin{document}

\section{Basic Components}

\subsection{Tin Bot} % ===================================

\subsubsection{Power LED}
Indicator LED for power.
\begin{description}
	\item[Failure Rate:] $\leq 10^{-5}$ per hour
	\item[Source:] Luminaire Lifetime report\footnote{\url{http://apps1.eere.energy.gov/buildings/publications/pdfs/ssl/led\_luminaire-lifetime-guide\_june2011.pdf}}
\end{description}

\subsubsection{Front LED}
Front LED on E-Puck.
\begin{description}
	\item[Failure Rate:] $\leq 10^{-5}$ per hour
	\item[Source:] Luminaire Lifetime report
\end{description}

\subsubsection{Victim Indicator LEDs}
Red LEDs on E-Puck.
\begin{description}
	\item[Failure Rate:] $\leq 10^{-5}$ per hour
	\item[Source:] Luminaire Lifetime report
\end{description}

\subsubsection{Power Switch}
E-Puck's power switch.
\begin{description}
	\item[Failure Rate:] $\leq 10^{-6}$ per hour
	\item[Source:] The failure rate is estimated to be low as the power switch is a simple standard component.
\end{description}

\subsubsection{IR-Sensors}
IR-sensors on the extension board.
\begin{description}
	\item[Failure Rate:] $\leq 10^{-5}$ per hour
	\item[Source:] Estimated to be similar to the LEDs.
\end{description}

\subsubsection{Proximity Sensors}
The E-Puck's proximity sensors.
\begin{description}
	\item[Failure Rate:] $\infty$
	\item[Source:] The proximity sensors do not seem to work as advertised.
	\item[Experiment Setup:] We sent the raw proximity sensor data via Bluetooth to a smartphone and saw that it clearly does not follow the specification: the sensor values decrease with the increasing distance and not even the operating range applies.
	\item[Desired Failure Rate:] $\leq 10^{-1}$ per hour
	\item[Source:] We have to do some more experiments and hope to achieve a better failure rate. For now we estimate a high failure rate of $\leq 10^{-1}$ to be able to give a result.
\end{description}

\subsubsection{ATmega328}
Processor controlling the signals picked up from the victim.
\begin{description}
	\item[Failure Rate:] $\leq 10^{-12}$ per hour
	\item[Source:] Atmel datasheet\footnote{\url{http://www.atmel.com/images/Atmel-8271-8-bit-AVR-Microcontroller-ATmega48A-48PA-88A-88PA-168A-168PA-328-328P\_datasheet\_Complete.pdf}}
\end{description}

\subsubsection{E-Puck Controller Board}
The E-Puck's controller.
\begin{description}
	\item[Failure Rate:] $\leq 10^{-10}$ per hour
	\item[Source:] The failure rate is estimated to be slightly higher than the one of ATmega328 because it has more components.
\end{description}

\subsubsection{E-Puck Motors and Driver Board}
The E-Puck's motors and motor controller.
\begin{description}
	\item[Failure Rate:] $\leq 10^{-5}$ per hour
	\item[Source:] The failure rate of the E-Puck's motors and motor controller is estimated to be low as it is a standard component and the E-Pucks are not assumed to be worn out.
\end{description}

\subsubsection{E-Puck Memory}
The E-Puck's memory.
\begin{description}
	\item[Failure Rate:] $\leq 10^{-11}$ per hour
	\item[Source:] We assume that the E-Puck's memory is comparable to the ATmega328's memory.\footnote{\url{http://www.atmel.com/images/Atmel-8271-8-bit-AVR-Microcontroller-ATmega48A-48PA-88A-88PA-168A-168PA-328-328P\_datasheet\_Complete.pdf}} \footnote{\url{https://www.wolframalpha.com/input/?i=\%281+hour\%29\%2F\%282e7+years\%29}}
\end{description}

\subsubsection{I2C Bus}
Cable connecting the controlling E-Puck and the E-Puck's extension board.
\begin{description}
	\item[Failure Rate:] $\leq 10^{-5}$ per hour
	\item[Source:] The failure rate is estimated to be low due to the simple construction.
\end{description}

\subsubsection{Magnets}
The magnets attached to the E-Puck.
    % Both positive and negative failures
\begin{description}
	\item[Failure Rate:] $\leq 10^{-5}$ per hour
	\item[Source:] The chance that the magnets trigger unintentionally, i.e. through a wall, is low since the Tin Bot has to drive close to a section of a wall while the victim's location coincidentally is close to this very wall.
\end{description}

\subsubsection{Bluetooth Module}
The E-Puck's Bluetooth module.
\begin{description}
	\item[Failure Rate:] $\leq 10^{-5}$ per hour
	\item[Source:] The failure rate of the Bluetooth module is estimated to be low as it is a standard component and the distances between the connected participants are small%
    \footnote{Example report about Bluetooth being reliable:
    \url{http://www.connectblue.com/fileadmin/Connectblue/Web2006/Documents/White\_papers/Bluetooth\_Reliability.pdf}}.
\end{description}

\subsection{Tin Bot Software} % ===================================

\subsubsection{Victim Recognition}
Module that determines whether we are carrying the victim.
\begin{description}
\item[Failure Rate:] $\leq 10^{-3}$ per hour
 %
\item[Source:] As this component uses an educated guess about angles,
size, and sensor sensitivity, this is estimated to be comparatively
unreliable.
 %
\end{description}

\subsubsection{Victim Triangulation}
 %
Module that triangulates the victim's position based on two or more
previous measurements.
 %
\begin{description}
\item[Failure Rate:] 0
 %
\item[Source:] Assuming correct implementation and correct automatic translation
to target hardware, the module works as intended.
 %
\end{description}

\subsubsection{Avoidance Watchdog}
 %
Module that shall interrupt the movement when we are about to crash into a wall.
 %
\begin{description}
\item[Failure Rate:] 0
 %
\item[Source:] Since the watchdog is a very simple component we assume a correct implementation and correct automatic translation to target hardware, therefore the module works as intended.
 %
\end{description}

\subsubsection{Crash Watchdog: Right-Hand Follower}
 %
The part of the Right-Hand Follower that tries to avoid crashing into obstacles.
 %
\begin{description}
\item[Failure Rate:] $\leq 10^{-1}$ per hour
 %
\item[Source:] We ran our virtual prototype of the Right-Hand Follower
for one simulated hour and did not observe any crash.
 %
\end{description}

\subsubsection{Crash Watchdog: Path Finder/Executor}
 %
The part of the Path Finder/Executor that tries to avoid crashing into obstacles.
 %
\begin{description}
\item[Failure Rate:] $\leq 10^{-2}$ per hour
 %
\item[Source:] We ran our virtual prototype of the Path Finder/Executor
for one simulated hour (with hand-chosen destinations upon completion) and
did not observe any crash.  We assume that the failure rate is less
than the one of the Right-Hand Follower as the update and check
mechanism is more sophisticated.
 %
\end{description}

\subsubsection{Path Pruning}
 %
Module that shall detect and determine when Right-Hand Follower is about to walk in circles.
 %
\begin{description}
\item[Failure Rate:] 0
 %
\item[Source:] Assuming correct implementation and correct automatic translation
to target hardware, the implementation is correct.
 %
\end{description}

\subsection{LPS} % ===================================

\subsubsection{Raspberry}
General Raspberry Pi module.
\begin{description}
\item[Failure Rate:] $\leq 10^{-6}$ per hour
\item[Source:] We approximate the reliability lower than the one of an FPGA ($10^{-8}$)\footnote{\url{http://www.xilinx.com/support/documentation/user_guides/ug116.pdf}}, but greater than the one of a Camera ($10^{-4}$).
\end{description}

\subsubsection{Camera}
Takes photo and converts it into the RAW format.
\begin{description}
\item[Failure Rate:] $\leq 10^{-4}$ per hour
\item[Source:] Educated guess based on the component's complexity.
\end{description}

\subsubsection{LPS Software (Ours)}
\begin{description}
\item[Failure Rate:] $0$
\item[Source:] Based on the fact that the software is only a loop
executing very simple code, we assume the code is correct.
\end{description}

\subsubsection{LPS Software (External Module)}
\begin{description}
\item[Failure Rate:] $0$
\item[Source:] We want to use well established libraries and perform rather simple computations, thus we assume this to be correct.
\end{description}

\subsubsection{LPS Operating System (Linux)}
The Raspberry Pi runs a Linux kernel and performs communications, as well as image processing tasks.
\begin{description}
\item[Failure Rate:] $\leq 2.94 \times 10^{-5}$ per hour
\item[Source:] The IBM Linux Technology Center conducted a 30 day stress test resulting in a success rate of 97.88\%\footnote{\url{https://www.ibm.com/developerworks/library/l-rel/}}.
\end{description}

\subsubsection{Power Supply}
\begin{description}
\item[Failure Rate:] $\leq 10^{-6}$
\item[Source:] Since the Raspberry Pi is widely used for long running processes such as server installation, we can safely assume that the power supply is rather reliable, thus we guess the failure rate educatedly.
\end{description}

\subsubsection{Electricity Grid}
\begin{description}
\item[Failure Rate:] $\leq 2.25 \times 10^{-5}$ %0,00002492389658
\item[Source:] Based on data by the Verband der Elektrotechnik Elektronik Informationstechnik (VDE) in 2014\footnote{\url{http://www.energiezukunft.eu/netze/netzausbau/deutsches-stromnetz-ist-aeusserst-zuverlaessig-gn103605/}}. 
We cannot assume that at the point of deployment there is no major environmental catastrophe taking place since we simulate a rescue team.
\end{description}

\subsection{Victim} % ===================================

\subsubsection{Battery}
The victim's main power source.
\begin{description}
\item[Failure Rate:] $\leq 10^{-8}$ per hour
\item[Source:] Based on the fact that this is a very simple component, the failure rate was estimated.
\end{description}

\subsubsection{ATtiny2313}
Processor controlling light emission of the victim.
\begin{description}
\item[Failure Rate:] $\leq 10^{-12}$ per hour
\item[Source:] Atmel datasheet for this microcontroller\footnote{\url{http://www.atmel.com/images/doc8246.pdf}}
\end{description}

\subsubsection{MOS-FET (IRLZ44N)}
Power amplification for IR-diodes.
\begin{description}
\item[Failure Rate:] $\leq 10^{-12}$ per hour
\item[Source:] Based on the fact that this is a very simple component, the failure rate was estimated.
\end{description}

\subsubsection{Resistor(s)}
\begin{description}
\item[Failure Rate:] $\leq 10^{-9}$ per hour
\item[Source:] Resistor Failure Rate model \footnote{\url{http://www.sqconline.com/resistor-failure-rate-model-mil-hdbk-217-rev-f-noetice-2}}.
\end{description}

\subsubsection{IR-LEDs}
\begin{description}
\item[Failure Rate:] $\leq 10^{-5}$ per hour
\item[Source:] Luminaire Lifetime report\footnote{\url{http://apps1.eere.energy.gov/buildings/publications/pdfs/ssl/led\_luminaire-lifetime-guide\_june2011.pdf}}, as it is assumed that IR LEDs (which use the same mechanism to emit photons) exhibit the same fault rate.
\end{description}

\subsubsection{Controller Software}
\begin{description}
\item[Failure Rate:] $0$
\item[Source:] Based on the fact that the software is only a loop
executing very simple code, we assume the code is correct.
\end{description}

\subsubsection{Magnet Belt}
\begin{description}
\item[Failure Rate:] $\leq 10^{-1}$ per hour
\item[Source:] Unreliable construction.
\end{description}


\subsection{Environment} % ===================================

\subsubsection{IR Medium}
 %
Measures how well the environment permits or perturbs an IR signal.
This also cares about noise and background \enquote{luminosity}.
 %
\begin{description}
\item[Failure Rate:] $\leq 10^{-4}$ per hour
 %
\item[Source:] As we assume that there is not much noise and background
luminosity, this should be reasonably small, and the IR signal of the
victim should remain mostly undisturbed by the environment.
 %
\end{description}

\subsubsection{Bluetooth Medium}
 %
Measures how well the environment permits or perturbs Bluetooth communication.
This also cares about noise and background \enquote{luminosity}.
 %
\begin{description}
\item[Failure Rate:] $\leq 10^{-3}$ per hour
 %
\item[Source:] As we assume that there is not much noise and background
luminosity, this should be reasonably small, and Bluetooth communication
of the victim should remain mostly undisturbed by the environment.
 %
\end{description}

 \pagebreak
\subsection*{Computed Failure Probabilities}

We chose $t = 1h$ for the computation of the failure probabilities, so
we have $F = 1-e^{-\lambda}$.  As many failure rates are repeated,
here is a conversion table:

%%%%%%  RAW VALUES
%0: 0.0
%0.1: 0.09516258196404043
%0.01: 0.009950166250831947
%1e-3: 0.0009995001666250085
%1e-4: 9.999500016666251e-05
%1e-05: 9.999950000166666e-06
%2.25e-05: 2.2499746876898428e-05
%2.94e-05: 2.9399567824235332e-05
%1e-06: 9.999995000001667e-07
%1e-09: 9.999999995e-10
%1e-10: 9.999999999500001e-11
%1e-11: 9.999999999949999e-12
%1e-12: 9.999999999995e-13

%%%%%%  UNTRUCATED/UNROUNDED VALUES
%\[\begin{array}{rr}
%0 & 0.0 \\
%0.1 & 0.09516258196404043 \\
%1\times 10^{-2} & 0.009950166250831947 \\
%1\times 10^{-3} & 0.0009995001666250085 \\
%1\times 10^{-4} & 9.999500016666251\times 10^{-5} \\
%1\times 10^{-5} & 9.999950000166666\times 10^{-6} \\
%2.25\times 10^{-5} & 2.2499746876898428\times 10^{-5} \\
%2.94\times 10^{-5} & 2.9399567824235332\times 10^{-5} \\
%1\times 10^{-6} & 9.999995000001667\times 10^{-7} \\
%1\times 10^{-9} & 9.999999995\times 10^{-10} \\
%1\times 10^{-10} & 9.999999999500001\times 10^{-11} \\
%1\times 10^{-11} & 9.999999999949999\times 10^{-12} \\
%1\times 10^{-12} & 9.999999999995\times 10^{-13} \\
%\end{array}\]

%%%%%%  ROUNDED VALUES (same rounding as in the trees)
\[\begin{array}{rr}
0.1                 & 0.095163 \\
1\times 10^{-2}     & 0.0099502 \\
1\times 10^{-3}     & 0.0009995 \\
1\times 10^{-4}     & 9.9995\times 10^{-5} \\
1\times 10^{-5}     & 1\times 10^{-5} \\
2.25\times 10^{-5}  & 2.25\times 10^{-5} \\
2.94\times 10^{-5}  & 2.94\times 10^{-5} \\
1\times 10^{-6}     & 1\times 10^{-6} \\
1\times 10^{-8}     & 1\times 10^{-8} \\
1\times 10^{-9}     & 1\times 10^{-9} \\
1\times 10^{-10}     & 1\times 10^{-10} \\
1\times 10^{-11}     & 1\times 10^{-11} \\
1\times 10^{-12}     & 1\times 10^{-12} \\
0                   & 0.0
\end{array}\]

Note that we used the built-in function \texttt{expm1} of Python to get more accurate
results than the naive approach.


\section{Fault Trees and Computed Failure Probabilities}

\subsection*{Top-Level Failures}

Note that you can click on the the node's pdf-file's name to open the respective fault tree.

Note that MR2, MR4, MR6, MR16, MR18, and MR19 do
not have fault trees, as this concept does not apply here.  Also note
that MR11 has been removed.

\newcommand{\refpdf}[1]{(\texttt{\href{trees/#1.pdf}{\texttt{#1.pdf}}})}

\begin{description}
\item[\texttt{power LED is off} \refpdf{nopowerled}:]
    The power LED is not lit. \\
    This means that either there is a failure in the E-Puck, or the power LED is faulty.
\item[\texttt{E-Puck failure}:]
    There is a problem with the E-Puck. 
    Either the user forgot to turn on the E-Puck or there is a problem with the hardware, such as a problem with the power, the board, or the memory.
\item[\texttt{power failure}:]
    The battery can be faulty, e.g. due to a leakage, or not charged. Additionally, the connection between the battery and its \enquote{consumer} might be impaired since a problem with the power switch prevents a connection.
\item[\texttt{uncooperative} \refpdf{uncooperative}:]
    The fault tree of behavior that appears to be uncooperative (MR3, MR22). \\
    The reason for such behavior is that the E-Puck does not know that there are other E-Puck or it did not receive any data to take into account because there was a problem with the communication protocol (T2T). 
    Alternatively, the transportation medium might prevent a meaningful communication.

    A problem with the receiver or the sender can also be the culprit, caused by a software bug, or a problem with the Bluetooth module. % we do not describe the subtrees separately, in regard to BW's and MS's conversation with HT
\item[\texttt{not using information about the victim} \refpdf{ignorevictim}:]
    The E-Puck ignores gathered information about the victim (MR5, MR13). \\
    Potential reasons for this behavior is that the extension board is faulty (see below), the triangulation failed, meaning that we effectively do not have meaningful data about the victim, there are inconsistencies in the data about the victim, or the E-Puck does not consume received information fast enough, such that it gets overwritten.
\item[\texttt{triangulation fails}:]
    Unless there is a software failure due to an unhandled edge case, the triangulation does not provide data because the E-Puck needs to travel a certain distance until a triangulation is triggered. 
    This distance is an empirically determined value and can be inappropriate. 
    Note that this distance is the length of a straight line orthogonal to the line of measure of older triangulations.
\item[\texttt{conflicting data about victim}:]
    Conflicting data about the victim can be the result of an unexpected movement by the victim, i.e. it is moved by the user. 
    Alternatively, uncooperative behavior might lead to conflicting data, since Tin Bots trust other Tin Bots unconditionally even if they send wrong data. 
    Additionally, faulty memory causes the same behavior.
\item[\texttt{run into walls} \refpdf{runintowall}:]
    A Tin Bot runs into a wall.\\
    The problem is either bad firmware, including that the wrong program has been uploaded to the E-Puck, or that the collision avoidance failed.
\item[\texttt{collision with obstacle}:]
    A collision can be the result of a problem in collision avoidance, or a failure to detect the obstacle.
\item[\texttt{obstacle not detected}:]
    Undetected obstacles are a result of walls that are not in conformity to the specification, e.g. hovering in the air or being too small. 
    Otherwise, a software failure or faulty proximity sensors can result in ignoring obstacles.
\item[\texttt{avoidance system fails}:]
    The avoidance system can only fail if any of the drivers fails to avoid a collision on its own \emph{and} the dedicated collision avoidance watchdog fails, too.
\item[\texttt{victim lost while escorting} \refpdf{victimlost}:]
    The fault tree of losing the victim during escort (MR7, MR8).\\
    This can happen if the victim was dropped since it was not grabbed properly, or the grabbing process failed entirely. Other than that, the victim might be pulled away by another Tin Bot.
\item[\texttt{victim dropped / unable to grab}:]
    For this to happen, the magnets can be too weak to properly grab and hold the victim. Alternatively, the belt might have fallen off.
\item[\texttt{pulled away by another Tin Bot}:]
    This can be a result of uncooperative behavior, e.g. one Tin Bot works overzealous or has not been informed, that the victim was already found. Moreover, the victim might have been picked up accidentally.
\item[\texttt{picking up the victim was accidental}:]
    Either the Tin Bot picked up the victim through a wall, which can happen since the walls do not shield the magnetic field, or there was actual physical contact involved.
\item[\texttt{picked up victim with physical contact}:]
    For this to happen, three events have to take place at the same time: The escort recognition fails, the Tin Bot did not detect an obstacle, and the magnets triggered unintentionally. This can only happen if the Tin Bot and the victim stood in an appropriate angle to each other by chance.
\item[\texttt{escort recognition fails}:]
    Either there is a problem in the SOS-Protocol, meaning the victim's signal was not sent or perceived properly, or the recognition method failed, e.g. the victim is in a blind spot during the escort.
\item[\texttt{SOS failure}:]
    If there is a signal, either the extension board failed, so the Tin Bot cannot pick up IR signals anymore, or the medium interferes with the signal transmission. If there is no signal, there is a failure in the victim.
\item[\texttt{obstacle not detected}:]
    Either the proximity sensor is faulty such that an obstacle cannot be recognized, or the walls do not meet the specification, e.g. they are not tall enough, or there is a software failure leading to signals being ignored.
\item[\texttt{not moving the victim out; at least not on the shortest known path} \refpdf{noescort}:]
    The fault tree of (apparently) not escorting the victim properly
    (MR8, MR17).\\
    There are three potential reasons for this: Firstly, the victim could have been picked up unintentionally (see respective tree's description). Secondly, the Tin Bot actually takes a detour during the escort. Lastly, the Tin Bot does not move at all.
\item[\texttt{taking a detour during escort}:]
    A detour can be a consequence of uncooperative behavior, e.g. another Tin Bot did not inform the escorting Tin Bot about potential shortcuts. Other than that, there might be a primary motor fault leading to unintentional movements. Lastly, a memory fault can trick a Tin Bot into talking suboptimal or even nonsensical paths.
\item[\texttt{escorting, but no LED indication} \refpdf{escortnoled}:]
    The fault tree of not flashing properly during escort (MR9). 
    Reasons for this behavior is that the Tin Bot is not aware of the fact that it is escorting the victim, the LED is faulty, or there is a software bug preventing the LED to be turned on.
\item[\texttt{not aware of escorting}:]
    Either the Tin Bot forgot about the victim due to a memory fault, or it picked up the victim unintentionally.
\item[\texttt{victim not found} \refpdf{victim404}:]
    Fault tree of not finding the victim (MR10).
     \\
    Unless the maze is outside the specified conditions (victim not in
    maze; unsolvable maze), the only reasons for which a Tin Bot can
    fail to find the victim (ignoring the low-level causes that are
    already handled elsewhere) are extension board failure (see below),
    victim signal failure (see below), losing orientation (see below),
    and driving in a circle (see next).
\item[\texttt{Right-Hand Follower drives in circles}:]
    A Tin Bot can only end up driving in a circle if the
    path-pruning\footnote{This is the module responsible for detecting
    and avoiding driving in circles.} fails, and the maze permits it
    (and is thus not 1-outerplanar, contrary to our requirement).  Note
    that the Right-Hand Follower itself does not contain logic that
    prevents it from driving in circles, thus its failure probability
    is 1.
\item[\texttt{Tin Bot lost orientation}:]
    In order to make a Tin Bot lose its orientation, first the LPS must
    be failing (see below), \emph{and} the approximation algorithm
    (including choice of all hard-coded constants) needs to be
    sufficiently far off, \emph{and} the drift\footnote{difference
    between real and perceived orientation and position} must not
    stabilize.
\item[\texttt{victim silent} \refpdf{victimsilent}:]
    Fault tree of the victim not sending any signal (MR22).
      \\
    The victim can only fail to send the specified signal if there is a
    hardware failure (see next), or a software bug.
\item[\texttt{victim hardware failure}:]
    Unless the victim's battery is not charged, this must be caused by
    some primary hardware failure (transistors, resistors, IR LEDs,
    battery).
\item[\texttt{clear line of sight, but no LED indication} \refpdf{seenoled}:]
    The fault tree of missing IR visualization (MR12, MR20, MR21).
     \\
    Unless the user did not turn on the E-Puck, the only causes for
    this are software bugs, hardware failure of the LED, extension
    board failure (see below), or a failure of the victim to send any
    signal (see previous).
\item[\texttt{standing still} \refpdf{standingstill}:]
    The fault tree of standing still (MR13).
     \\
    Unless the setup is bad (e.g.\ wrong firmware), this is either
    caused by a severe software failure (bug), picking up the victim
    through a wall (see above, \enquote{accidental pickup}),
    having run into a wall (see below),
    inability to turn wheels (see below), general E-Puck failure (see
    below), or a failure to poll the initial position from the LPS.
\item[\texttt{wheels do not turn}:]
    Unless the wheels are blocked (secondary failure), this can only be
    due to a primary motor fault. (Note that faults like battery faults etc.\
    are handled in the general E-Puck failure.)
\item[\texttt{E-Puck failure}:]
    This can be due to hardware failures (memory, controller board,
    controller), user errors (not switched on), or a power failure.
\item[\texttt{E-Puck power failure}:]
    A power failure can only be due to hardware faults (power switch,
    battery) or user error (battery not charged).
\item[\texttt{no initial position from LPS}:]
    Unless the user did not enable the LPS as specified, this 
    failure is either due to the Bluetooth link being down, or the LPS
    sending no data.
\item[\texttt{LPP link down}:]
    The only faults that can take down the LPP link are hardware
    failures in the receiver or the medium, or a simulated LPS outage.
    Note that a sender failure is already handled in the more general
    failure of not getting an initial position from the LPS (see
    previous).
\item[\texttt{no initial position from LPS}:]
    This can only happen if there is a general LPS failure from the beginning.
\item[\texttt{LPS failure}:]
    The LPS only sends no data if there was a failure in the Color Blob
    Protocol (which is either due to a camera fault, or
    outside-specification conditions), or the Raspberry Pi is failing.
    The only causes for this are hardware failures (e.g.\ power)
    and software failures (bug in any of OS, external module, or
    our module).
\item[\texttt{moving to the \enquote{gathered position}} \refpdf{gowrong}:]
    The fault tree of going to the wrong location at a specific step
    during the rescue action (MR14).
     \\
    This can only happen if there was a grave misunderstanding about
    MR14 \emph{and} this misunderstanding was not discovered during
    peer review.
\item[\texttt{run into walls} \refpdf{runintowall}:]
    The fault tree of colliding with a wall or unintentionally with a
    victim (MR15).
     \\
    We can only run into a wall if the wrong firmware was uploaded to
    the E-Puck, the wall (generalized: obstacle) was not detected (see
    next), or the avoidance system failed (see above).
\item[\texttt{obstacle not detected}:]
    Unless the obstacle/wall was outside our specification, the only
    reasons to fail at detecting an obstacle are faulty proximity
    sensors, and primary software fault (e.g.\ the module that should
    ignore the victim's proximity data accidentally ignores a wall).
\item[\texttt{spurious movements} \refpdf{spuriousmovements}:]
    The fault tree of spurious or unreasonable movements (MR23 and
    several others).
     \\
    There are several possible causes for spurious movements.  It could
    be caused by a software bug, or a primary proximity sensor fault.
    It can also be due to a secondary setup failure: having the wrong
    color on top of the Tin Bot leads to mis-identification.  The
    previously mentioned LPS failure and uncooperative behavior (see
    above each) are also possible reasons. Finally, this is also a way
    to describe the failure to stop turning.
\item[\texttt{does not stop turning}:]
    This can be due to the victim not sending a valid signal (see
    above) or due to an extension board failure (see next).
\item[\texttt{extension board failure}:]
    This is the generalization of any failure in the microcontroller,
    IR sensor, I2C bus, or extension board circuit.
\item[\texttt{system failure} \refpdf{systemfailure}:]
    The overall fault tree of the victim not being moved out.  Note
    that this can only happen if \emph{all} Tin Bots fail to move the
    victim out, and is thus a disjunction of the following, previously
    mentioned top-level failures that are fatal to the Tin Bot's
    ability to operate:
     \\
    Victim cannot be found, not using information
    about the victim, not moving the victim out, and
    victim lost while escorting.
\end{description}



\end{document}
