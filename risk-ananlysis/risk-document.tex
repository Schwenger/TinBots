\documentclass[a4paper,parskip,headheight=38pt]{scrartcl} % article or scrartcl
\usepackage[utf8]{inputenc}
\usepackage[T1]{fontenc}
\usepackage{amsmath,amssymb,amsfonts}
\usepackage[%
  automark,
  headsepline                %% Separation line below the header
]{scrlayer-scrpage}
\usepackage[english]{babel}
\usepackage{hyphenat}
\usepackage[hidelinks]{hyperref}
\usepackage[top=1.4in, bottom=1.5in, left=1in, right=1in]{geometry}
\usepackage{lastpage}
\usepackage{csquotes}
\usepackage{microtype}
\usepackage{datetime}

\usepackage[normalem]{ulem}
\usepackage{enumerate}
\usepackage{hyperref}

\usepackage{microtype}

\usepackage[hang]{footmisc}
\setlength{\footnotemargin}{3mm}

\usepackage{enumitem}

\usepackage[export]{adjustbox}

% \usepackage{multicol}
\usepackage{graphicx}
\usepackage{graphics}
% \usepackage{float}
% \usepackage{caption}

\parindent 0pt
\parskip 6pt

\clubpenalty = 10000
\widowpenalty = 10000
\displaywidowpenalty = 10000

\setkomafont{pagehead}{\normalfont\sffamily\footnotesize}
\addtolength{\headheight}{+13pt}
\lohead{Marlene Böhmer, s9meboeh@stud.uni-saarland.de, 2547718 \\
	Maximilian Köhl, s8makoeh@stud.uni-saarland.de, 2553525 \\
	Maximilian Schwenger, schwenger@stud.uni-saarland.de, 2542438\\
	Ben Wiederhake, s9bewied@stud.uni-saarland.de, 2541266}
\rohead{\includegraphics[height=36pt, right]{../logo/logo.png} \newline ES16, Risk Document, Group 6, Page {\thepage}/{\pageref*{LastPage}}}

\newtimeformat{mytime}{\twodigit{\THEHOUR}\twodigit{\THEMINUTE}\twodigit{\THESECOND}}
\settimeformat{mytime}
\newdateformat{mydate}{\twodigit{\THEYEAR}\twodigit{\THEMONTH}\twodigit{\THEDAY}}
\cfoot{\tiny\texttt{ID \mydate\today\currenttime}}
\chead{} % Needed because now the \subsections get displayed
\pagestyle{scrheadings}

% \renewcommand{\headrulewidth}{0pt}
% \addtolength{\textheight}{+30mm}
% \addtolength{\textwidth}{+50mm}
% \addtolength{\hoffset}{-7mm}

% \newcommand{\Omicron}{\ensuremath{\mathcal{O}}}
% \newcommand{\omicron}{\ensuremath{o}}
% \newcommand{\set}[1]{\{#1\}}
% \newcommand{\abs}[1]{\lvert #1 \rvert}

\DeclareMathOperator{\sinc}{sinc}

\newcommand{\incomplete}[1]{\textless{} #1 \textgreater{}}

\begin{document}

\section{Basic Components}
\subsection{Tin Bot} % ===================================

\subsubsection{Power LED}
\incomplete{FIXME}

\subsubsection{Front LED}
\incomplete{FIXME}

\subsubsection{Victim Indicator LEDs}
\incomplete{FIXME}

\subsubsection{Power Switch}
\incomplete{FIXME}

\subsubsection{IR-Sensors}
\incomplete{FIXME}

\subsubsection{Proximity Sensors}
\incomplete{FIXME}

\subsubsection{ATmega328}
\incomplete{FIXME}

\subsubsection{E-Puck Controller Board}
\incomplete{FIXME}

\subsubsection{E-Puck Motors and Driver Board}
\incomplete{FIXME}

\subsubsection{E-Puck Memory}
\incomplete{FIXME}

\subsubsection{I2C Bus}
\incomplete{FIXME}

\subsubsection{Magnets}
    % Both positive and negative failures
\incomplete{FIXME}

\subsubsection{Bluetooth Module}
\incomplete{FIXME}

\subsection{Tin Bot Software} % ===================================

\subsubsection{Victim Recognition}
    % "Is there a victim sticking to us in this moment?"
    % Both false positive and false negative
\incomplete{FIXME}

\subsubsection{Victim Triangulation}
    % Probability for "too high min-distance\nfor sensing again"
\incomplete{FIXME}

\subsubsection{Avoidance Watchdog}
\incomplete{FIXME}

\subsubsection{Path Pruning}
\incomplete{FIXME}

    % "Don't needlessly walk in circles"

\subsubsection{Right-Hand Follower} % can it collide?
\incomplete{FIXME}

\subsubsection{Path Finder/Executor} % can it collide?
\incomplete{FIXME}

\subsection{LPS} % ===================================

\subsubsection{Raspberry}
\incomplete{FIXME}

\subsubsection{Camera}
\incomplete{FIXME}

\subsubsection{USB Connection}
\incomplete{FIXME}

\subsubsection{LPS Software (Ours)}
\incomplete{FIXME}

\subsubsection{LPS Software (External Module)}
\incomplete{FIXME}

\subsubsection{LPS Operating System (Linux)}
\incomplete{FIXME}

\subsubsection{Power Supply}
\incomplete{FIXME}

\subsubsection{Electricity Grid}
\incomplete{FIXME}

\subsection{Victim} % ===================================

\subsubsection{Battery}
Main powers source of the Victim.
\begin{description}
\item[Failure Rate:] $\leq 10^{-6}$ per hour
\item[Source:] Based on the fact that this is a very simple component and is not used frequently, the probability was estimated.
\end{description}

\subsubsection{ATtiny2313}
\incomplete{FIXME}

\subsubsection{MOS-FET (IRLZ44N)}
\incomplete{FIXME}

\subsubsection{Resistor(s)}
\incomplete{FIXME}

\subsubsection{IR-LEDs}
\incomplete{FIXME}

\subsubsection{Controller Software}
\incomplete{FIXME}

\subsubsection{Magnet Belt}
\incomplete{FIXME}

\subsection{Environment} % ===================================

\subsubsection{IR Medium}
\incomplete{FIXME}

\subsubsection{Bluetooth Medium}

\incomplete{FIXME}


\section{Fault Trees and Computed Failure Probabilities}

Please note that due to the complex interactions of the
four\footnote{LPS, E-Puck, extension board, victim} autonomous systems,
our fault tree\footnote{technically: fault DAG, as some intermediate
failures share cause fault/failure} is very large.  Thus, it is not
meaningful to write a paragraph about each intermediate failure.
Instead, we chose to leave out \emph{some} of the intermediate
failures.

Please note that \emph{all} the top-level failures and \emph{all} basic
components are listed.

% Example doesn't use any numeration, so we don't, either.
\subsection*{Top-Level \texttt{runintowall}-Failure and Intermediate Events and their Probabilites}

\incomplete{in-depth explanation of the file \texttt{runintowall.pdf}}

\incomplete{FIXME}


\subsection*{Other Top-Level Failures}

Note that MR2, MR4, MR6, MR16, MR18, and MR19 do
not have fault trees, as this concept does not apply here.  Also note
that MR11 has been removed.

\begin{description}
\item[\texttt{nopowerled}:]
    The fault tree of the power LED staying off (MR1).
\item[\texttt{uncooperative}:]
    The fault tree of behavior that appears to be uncooperative (MR3, MR22).
\item[\texttt{ignorevictim}:]
    The fault tree of apparently ignoring information about the victim
    (MR5, MR13).
\item[\texttt{victimlost}:]
    The fault tree of losing the victim during escort (MR7, MR8).
\item[\texttt{noescort}:]
    The fault tree of (apparently) not escorting the victim properly
    (MR8, MR17).
\item[\texttt{escortnoled}:]
    The fault tree of not flashing properly during escort (MR9).
\item[\texttt{victim404}, \texttt{victimsilent}:]
    Fault trees of not finding the victim (MR10).
\item[\texttt{seenoled}:]
    The fault tree of the IR visualization missing (MR12, MR20, MR21).
\item[\texttt{standingstill}:]
    The fault tree of standing still (MR13).
\item[\texttt{gowrong}:]
    The fault tree of going to the wrong location at a specific step
    during the rescue action (MR14).
\item[\texttt{runintowall}:]
    The fault tree of colliding with a wall or unintentionally with a
    victim (MR15).
\item[\texttt{spuriousmovements}:]
    The fault tree of spurious or unreasonable movements (MR23 and
    several others).
\item[\texttt{systemfailure}:]
    The overall fault tree of the victim not being moved out.  Note
    that this is essentially the \texttt{OR} of some\footnote{e.g.\
    \texttt{standingstill}, but not \texttt{seenoled}} other fault
    trees.
\end{description}


\end{document}
