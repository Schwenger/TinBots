\documentclass[a4paper,parskip,headheight=38pt]{scrartcl} % article or scrartcl
\usepackage[utf8]{inputenc}
\usepackage[T1]{fontenc}
\usepackage{amsmath,amssymb,amsfonts}
\usepackage[%
  automark,
  headsepline                %% Separation line below the header
]{scrlayer-scrpage}
\usepackage[english]{babel}
\usepackage{hyphenat}
\usepackage[hidelinks]{hyperref}
\usepackage[top=1.4in, bottom=1.5in, left=1in, right=1in]{geometry}
\usepackage{lastpage}
\usepackage{csquotes}
\usepackage{microtype}
\usepackage{datetime}

\usepackage[normalem]{ulem}
\usepackage{enumerate}
\usepackage{hyperref}

% \usepackage{multicol}
\usepackage{graphicx}
\usepackage{graphics}
% \usepackage{float}
% \usepackage{caption}

\setkomafont{pagehead}{\normalfont\sffamily\footnotesize}
\addtolength{\headheight}{+6pt}
\lohead{Marlene Böhmer, s9meboeh@stud\ldots, 2547718 \\
	Maximilian Köhl, mail@koehlma.de, 2553525 \\
	Ben Wiederhake, s9bewied@stud\ldots, 2541266}
\rohead{\newline \newline ES16, Set 2, Page {\thepage}/{\pageref*{LastPage}}}

\newtimeformat{mytime}{\twodigit{\THEHOUR}\twodigit{\THEMINUTE}\twodigit{\THESECOND}}
\settimeformat{mytime}
\newdateformat{mydate}{\twodigit{\THEYEAR}\twodigit{\THEMONTH}\twodigit{\THEDAY}}
\cfoot{\tiny\texttt{ID \mydate\today\currenttime}}
\chead{} % Needed because now the \subsections get displayed
\pagestyle{scrheadings}

% \renewcommand{\headrulewidth}{0pt}
% \addtolength{\textheight}{+30mm}
% \addtolength{\textwidth}{+50mm}
% \addtolength{\hoffset}{-7mm}

% \newcommand{\Omicron}{\ensuremath{\mathcal{O}}}
% \newcommand{\omicron}{\ensuremath{o}}
% \newcommand{\set}[1]{\{#1\}}
% \newcommand{\abs}[1]{\lvert #1 \rvert}

\DeclareMathOperator{\sinc}{sinc}

\begin{document}

\section{Overview}
A set of robots that survey a simulated collapsed building (maze), try to find a victim (teddy), and rescue it.

\section{Functional Requirements}
\subsection*{A: Must-Have}
\begin{description}
\item[A1] while the Tin Bot is operational a green LED shall be on
\item[A1] the \enquote{LPS} (local positioning system) needs to supply the E-Pucks with location and orientation data
\item[A1] the E-Pucks must share information about the victim
\item[A1] cartograph the inspected environment using the given LPS signal
\item[A1] find the victim by…
\item[A1.1] … exhaustive search, if nothing about the victim is known
\item[A1.1] … attempted shortest path, if the location of the victim is known
\item[A1] bring the victim out
\item[A1] while the Tin Bot is escorting the victim a red LED shall be on
\item[A1] use the ir-sensors to pick up signals from the victim
\end{description}

\subsection*{B: Nice-to-Have}
\begin{description}
\item[B1] be able to behave reasonably if the LPS is missing
\item[B1] export the map
\item[B1] be able to behave reasonably if some of the E-Pucks go defunct
\item[B1] be able to bahave reasonably if the map changes
\item[B1] detect if there is no victim
\end{description}

\subsection*{C: Must-not-Have}
\begin{description}
\item[C1] use actual GPS
\item[C2] use actual gyroscope
\item[C3] do not use camara to identify the victim
\item[C4] unsolvable mazes
\item[C5] deal with more than one victim
\item[C6] do not try to remove obstacles
\item[C7] do not try to do detect / react to a malfunctioning / byzantine LPS (although a certain tolerance must be respected of course)
\end{description}

\section{Non-Functional Requirements}
\subsection*{A: Must-Have}
\begin{description}
\item[A1] must not be slower than worst-case brute force
\item[A2] if there is information do not just run to the position where the information was gathered but instead try to actually localize and find the victim
\item[A3] avoid any collisions except with the victim
\item[A4] do not make any assumptions about the maze except being solvable
\item[A5] the exists of the building are the set of the starting positions of the E-Pucks
\end{description}

\subsection*{B: Nice-to-Have}
\begin{description}
\item[B1] be gentle
\item[B2] use the shortest kown path out of the building (see starting positions)
\end{description}

\subsection*{C: Must-not-Have}
\begin{description}
\item[C1] deal with uneven floors or multiple floor buildings
\item[C2] use actually collapsed buildings or victims
\end{description}


\section{Use-Cases}
\subsection{System Startup}
primary actor: user \\
goal in context: activate the E-Puck(s) \\
precondition: LPS is up and running, E-Puck-batteries are connected and loaded up, E-Puck power switches are in the off position. \\
trigger: power switch of 1 or more E-Pucks is switched on \\
scenario: \\
1. user places E-Pucks on surface \\
2. user switches on the LPS and waits a second \\
3. user switches on the E-Pucks \\
exceptions: E-Pucks are too far away; LPS is too far away; E-Pucks cannot properly move on it's own (e.g. placed upside-down) \\

\subsection{System Shutdown}
primary actor: user \\
goal in context: deactive the simulation / system \\
precondition: At least on E-Puck's power switch is in the on position \\
trigger: power switch of the currently-switched-on E-Pucks are switched off \\
scenario: \\
1. user switches off some of the running E-Pucks \\
2. user switches off the LPS \\
3. user switches off the remaining E-Pucks \\

\subsection{Removing an E-Puck}
primary actor: FIXME \\
goal in context: FIXME \\
precondition: FIXME \\
trigger: FIXME \\
scenario: \\
1. FIXME \\
exceptions: FIXME \\

\subsection{Adding/Removing Walls}
primary actor: FIXME \\
goal in context: FIXME \\
precondition: FIXME \\
trigger: FIXME \\
scenario: \\
1. FIXME \\
exceptions: FIXME \\

\subsection{Disaling LPS}
primary actor: FIXME \\
goal in context: FIXME \\
precondition: FIXME \\
trigger: FIXME \\
scenario: \\
1. FIXME \\
exceptions: FIXME \\

\subsection{Re-enabling LPS}
primary actor: FIXME \\
goal in context: FIXME \\
precondition: FIXME \\
trigger: FIXME \\
scenario: \\
1. FIXME \\
exceptions: FIXME \\


\end{document}
