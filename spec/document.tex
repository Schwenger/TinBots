\documentclass[a4paper,parskip,headheight=38pt]{scrartcl} % article or scrartcl
\usepackage[utf8]{inputenc}
\usepackage[T1]{fontenc}
\usepackage{amsmath,amssymb,amsfonts}
\usepackage[%
  automark,
  headsepline                %% Separation line below the header
]{scrlayer-scrpage}
\usepackage[english]{babel}
\usepackage{hyphenat}
\usepackage[hidelinks]{hyperref}
\usepackage[top=1.4in, bottom=1.5in, left=1in, right=1in]{geometry}
\usepackage{lastpage}
\usepackage{csquotes}
\usepackage{microtype}
\usepackage{datetime}

\usepackage[normalem]{ulem}
\usepackage{enumerate}
\usepackage{hyperref}

\usepackage{microtype}

\usepackage[hang]{footmisc}
\setlength{\footnotemargin}{3mm}

\usepackage{enumitem}

\usepackage[export]{adjustbox}

% \usepackage{multicol}
\usepackage{graphicx}
\usepackage{graphics}
% \usepackage{float}
% \usepackage{caption}

\parindent 0pt
\parskip 6pt

\clubpenalty = 10000
\widowpenalty = 10000
\displaywidowpenalty = 10000

\renewcommand\thesubsection{\Alph{subsection}}

\setkomafont{pagehead}{\normalfont\sffamily\footnotesize}
\addtolength{\headheight}{+13pt}
\lohead{Marlene Böhmer, s9meboeh@stud.uni-saarland.de, 2547718 \\
	Maximilian Köhl, s8makoeh@stud.uni-saarland.de, 2553525 \\
	Maximilian Schwenger, schwenger@stud.uni-saarland.de, 2542438\\
	Ben Wiederhake, s9bewied@stud.uni-saarland.de, 2541266}
\rohead{\includegraphics[height=36pt, right]{../logo/logo.png} \newline ES16, Specification, Group 6, Page {\thepage}/{\pageref*{LastPage}}}

\newtimeformat{mytime}{\twodigit{\THEHOUR}\twodigit{\THEMINUTE}\twodigit{\THESECOND}}
\settimeformat{mytime}
\newdateformat{mydate}{\twodigit{\THEYEAR}\twodigit{\THEMONTH}\twodigit{\THEDAY}}
\cfoot{\tiny\texttt{ID \mydate\today\currenttime}}
\chead{} % Needed because now the \subsections get displayed
\pagestyle{scrheadings}

% \renewcommand{\headrulewidth}{0pt}
% \addtolength{\textheight}{+30mm}
% \addtolength{\textwidth}{+50mm}
% \addtolength{\hoffset}{-7mm}

% \newcommand{\Omicron}{\ensuremath{\mathcal{O}}}
% \newcommand{\omicron}{\ensuremath{o}}
% \newcommand{\set}[1]{\{#1\}}
% \newcommand{\abs}[1]{\lvert #1 \rvert}

\DeclareMathOperator{\sinc}{sinc}

\begin{document}


\setlist[enumerate]{leftmargin=50pt}

\newcommand{\defkind}[2]{
	\newcounter{counter#2}
    \addtocounter{counter#2}{1}
	\newcommand{#1}{{\textbf{{#2}\arabic{counter#2}\addtocounter{counter#2}{1}}}}
}

\defkind{\musthave}{MR}
\defkind{\nicetohave}{NR}
\defkind{\mustnothave}{N}
\defkind{\plant}{E}
\defkind{\sensors}{S}
\defkind{\actuators}{A}
\defkind{\hardware}{H}
\defkind{\libs}{L}




\section{Overview}
After a major catastrophe a brave team of robots is sent out to save a damsel in distress. The collapsed building is simulated by a maze in which the victim, represented by a teddy bear, waits for it's rescue. The maze is laid out on a table, the walls are built out of paper. Some of the walls contain holes as windows or broken down beams of wood or steel. Will they save the day?

\section{Functional Requirements}
\subsection{Must-Have}
\begin{enumerate}[label=\musthave]
\item while the Tin Bot is operational, a green LED shall be on
\item the \enquote{LPS} (local positioning system) needs to supply the Tin Bots with location and orientation data via Bluetooth
\item the Tin Bots must share information about the victim
\item chart the inspected environment using the given LPS signal
\item find the victim by…
\begin{enumerate}
\item … exhaustive search, if nothing about the victim's location is known
\item … attempted shortest path, if information about the victim's location is available
\end{enumerate}
\item grab the victim using magnets after finding them
\item use proximity sensors / IR receivers to detect whether victim is grabbed correctly
\item move the victim out
\item while the Tin Bot is escorting the victim, the front red LED shall be flashing
\item use the IR sensors to pick up signals from the victim
\item detect and stay within the borders of the table
\item the surrounding 8 red LEDs shall represent which/whether the IR sensors receive some signal
\end{enumerate}

\subsection{Nice-to-Have}
\begin{enumerate}[label=\nicetohave,ref=\nicetohave]
\item be able to behave reasonably in terms of continuing the search if the LPS goes defunct\label{req:missLPS}
\item export the map
\item be able to behave reasonably if some of the Tin Bots go defunct \label{req:puckFault}
\item be able to behave reasonably if the map changes \label{req:mapChange}
\item detect if there is no victim
\end{enumerate}

\subsection{Must-not-Have}
\begin{enumerate}[label=\mustnothave]
\item use actual GPS
\item use actual gyroscope
\item use camera to identify the victim
\item deal with unsolvable mazes
\item deal with more than one victim
\item try to remove obstacles with Tin Bots
\item try to detect / react to a malfunctioning / byzantine LPS (although a certain tolerance must be respected)
\item secure against any kind of attack (physical or non physical)
\end{enumerate}

\section{Non-Functional Requirements}
\subsection{Must-Have}
\begin{enumerate}[label=\musthave]
\item must not be slower than worst-case brute force
\item if there is information do not just run to the position where the information was gathered but instead try to actually localize and find the victim efficiently
\item avoid any collisions except with the victim
\item do not make any assumptions about the maze except being solvable
\item the exits of the building are the set of the Tin Bots' starting positions
\item Tin Bots are equipped with a visual marker on top
\item the custom extension boards (with IR sensors and ATmega328p) are easily deployable / removable
\item sampling sensor information from IR-Sensors in a real time manner using the ATmega328p 
\item pick up received signals from ATmega328p in a real time manner such that no information is lost by being overwritten
\item transmitting pre-specified time-critical signal sequence using the IR-emitters using the ATtiny2313
\item each ATmega328p must be aware of the color it is \enquote{wearing} (for self-identification via the LPS)
\end{enumerate}

\subsection{Nice-to-Have}
\begin{enumerate}[label=\nicetohave]
\item be gentle to the victim, i.e. do not knock them over or squeeze against a wall
\item use the shortest known path out of the building (see starting positions)
\end{enumerate}

\subsection{Must-not-Have}
\begin{enumerate}[label=\mustnothave]
\item deal with uneven floors or multiple floor buildings
\item use actual collapsed buildings or victims
\item obstacles that are not detected by the proximity sensors (e.g., walls that are too small)
\item deal with more than 8 Tin Bots
\end{enumerate}


\section{Use-Cases}
\subsection{System Startup}
\begin{description}
\item[primary actor:] user
\item[goal in context:] activate the Tin Bot(s)
\item[precondition:] LPS is up and running; E-Puck-batteries are connected and charged up;
  Tin Bot power switches are in the off position; our custom extension boards on the
  E-Pucks % don't change to Tin Bot!
  are equipped and clearly visible from the LPS; the custom extension boards have
  pairwise distinct colors (for identification by the LPS)
\item[trigger:] power switch of one or more Tin Bots is switched on
\item[scenario:] \ 
\begin{enumerate}
	\item user places Tin Bots on surface
	\item user switches on the Tin Bots
\end{enumerate}
\item[exceptions:] Tin Bots are too far away; LPS is too far away; any Tin Bot cannot properly move on its own (e.g. placed upside-down)
\end{description}

\subsection{System Shutdown}
\begin{description}
\item[primary actor:] user
\item[goal in context:] deactivate the simulation / system
\item[precondition:] at least one Tin Bot's power switch is in the on position
\item[trigger:] power switches of the currently-switched-on Tin Bots are switched off
\item[scenario:] \ 
\begin{enumerate}
	\item user switches off all of the running Tin Bots
	\item user switches off the LPS
\end{enumerate}
\end{description}

\subsection{Removing A Tin Bot}
\begin{description}
\item[primary actor:] user
\item[goal in context:] simulate hardware failure in one of the Tin Bots
\item[precondition:] at least two Tin Bots one the table are switched on and running; requirement \ref{req:puckFault} is implemented
\item[trigger:] power switch of one of the switched-on Tin Bots is switched off
\item[scenario:] \ 
\begin{enumerate}
	\item user selects some Tin Bots to be deactivated (at most all except one)
	\item user switches off these Tin Bots
	\item user removes none, some, or all of these Tin Bots (the other switched-off Tin Bots shall be considered \enquote{obstacles})
\end{enumerate}
\item[exceptions:] only one switched-on Tin Bot remains; no Tin Bots are running; switching off and not removing any of the selected Tin Bots would render the maze unsolvable
\end{description}

\subsection{Adding/Removing Walls}
\begin{description}
\item[primary actor:] user
\item[goal in context:] simulate walls segment that collapse or break away
\item[precondition:] requirement \ref{req:mapChange} is implemented; E-Pucks are running; if removing a wall segment: at least one \enquote{wall}-segment on the table acting as a wall segment
\item[trigger:] user manually removes or adds one or more of the wall segment
\item[scenario:] \ 
\begin{enumerate}
	\item user removes some wall segment
	\item user places the removed wall segment somewhere else
\end{enumerate}
\item[exceptions:] adding a wall must not render the maze unsolvable
\end{description}

\subsection{Disabling LPS}
\begin{description}
\item[primary actor:] user
\item[goal in context:] simulate connectivity problems between LPS and Tin Bots
\item[precondition:] LPS is up and running; requirement \ref{req:missLPS} is implemented
\item[trigger:] user switches off the LPS (either by software, or by switching off the Raspberry Pi)
\item[scenario:] \ 
\begin{enumerate}
	\item user literally switches off the Raspberry Pi
\end{enumerate}
\end{description}

\subsection{Re-enabling LPS}
\begin{description}
\item[primary actor:] user
\item[goal in context:] simulate LPS coming back up again (see previous use case)
\item[precondition:] LPS is not running
\item[trigger:] user undoes the action from the disabling step
\item[scenario:]\
\begin{enumerate}
	\item user switches the Raspberry Pi back on
\end{enumerate}
\end{description}

\section{Environment properties}

\subsection{Plant}
\begin{enumerate}[label=\plant]
\item victim placed in maze
\item victim sending signals (using IR emitters)
\item victim wears magnetic belt
\item solvable maze (including table and walls; walls need to be of sufficient height to prevent signals radiating over the walls)
\item flat surface
\item unblocked view from the camera to the maze
\end{enumerate}

\subsection{Sensors}
\begin{enumerate}[label=\sensors]
\item power switches
\item IR receivers
\item proximity sensors
\item Bluetooth receiver
\item camera (LPS)s
\item ground sensors
\end{enumerate}

\subsection{Actuators}
\begin{enumerate}[label=\actuators]
\item Bluetooth emitters
\item motors (two per E-Puck)
\item IR emitters (victim)
\item LEDs (E-Pucks)
\end{enumerate}


\section{Platform}

\subsection{Hardware}
\begin{enumerate}[label=\hardware]
\item at least two E-Pucks
\item ATmega328p (to connect IR sensor with E-Pucks)
\item IR sensors (for E-Pucks)
\item magnets (for E-Pucks)
\item Raspberry Pi 3 (for LPS)
\item camera (including stand) (for LPS)
\item teddy (a.k.a. victim)
\item magnet belt (for teddy)
\item IR emitter (for teddy)
\item ATtiny2313 (to put unique signal on IR emitter)
\end{enumerate}

\subsection{Libraries}
\begin{enumerate}[label=\libs]
\item E-Puck base libraries
\item Raspberry Pi base system (Linux, \ldots)
\item Raspberry Pi Camera module
\item OpenCV or similar (for image analysis)
\item base libraries for ATmega328p and ATtiny2313
\end{enumerate}

\end{document}
